\LTcapwidth=\textwidth
\bibpunct{}{}{;}{a}{}{;}
%\singlespacing
\begin{small}
\begin{longtable}{>{\RaggedRight}p{3.6cm}>{\RaggedRight}p{7.3cm}>{\RaggedRight}p{3.6cm}}

\caption{Selected ecological theory relevant to ecological portfolios.}\\

\toprule

\textbf{Ecological theory} &
\textbf{Relevance to ecological portfolios} &
\textbf{Selected references} \\

\midrule
\multicolumn{2}{l}{\textbf{Sources of portfolio structure}} \\
\midrule

Community ecology &
We can represent species as assets and a community as a portfolio. Species interactions such as predation and competition may complicate the analogy. &
\citep{figge2004, morin2011}\\

Theory of island biogeography &
Explains a source of diversification for ``islands'', which have a portfolio-like structure. Larger islands may have higher levels of portfolio diversification. &
\citep{macarthur1967}\\

Metapopulations &
Subpopulations can act as diverse ecological assets as part of a metapopulation portfolio. Metapopulations usually have exchange between subpopulations (ecological assets), which may not have an analogy in financial portfolios. &
\citep{levins1969}\\

Functional groups &
Species that perform separate ecological functions can form ecological assets as part of an ecological portfolio. &
\citep{walker1992, thibaut2012}\\

\midrule
\multicolumn{2}{l}{\textbf{Causes of diversification and portfolio dynamics}}\\
\midrule

Moran effect &
Identifies that similar environments will induce similar population responses decreasing the diversity of a portfolio. Ecological assets that are further apart are expected to provide greater diversification. &
\citep{moran1949, ranta1998}\\

\bibpunct{(}{)}{;}{a}{}{} % add back brackets for in-text citation
Synchrony &
Often loosely used to referred to as correlation between ecological assets. Defined quantitatively by \citet{loreau2008} as a diversity-independent metric of correlation. The benefit of diversification is greater when synchrony of ecological assets is low. &%
\bibpunct{}{}{;}{a}{}{}%
\citep{ranta1998,moore2010,yeakel2014}\\

Unified neutral theory of biodiversity and biogeography &
Asks what community dynamics we would observe if species were functionally equivalent. We can think of a neutral community with ecological equivalence as an un-diversified portfolio compared to a diversified portfolio in which species show functional diversity. &
\citep{hubbell2001}\\

Biocomplexity &
Identifies that subpopulations can display a range of biological traits and behaviours and that these ranges can form portfolio diversification &
\citep{hilborn2003, hutchinson2008}\\

Response diversity &
Identifies elements of ecological systems that cause them to respond differently to perturbation. Potentially important aspect of ensuring stable portfolios as the magnitude and frequency of environmental stressors increase (e.g.\ climate change). &
\citep{elmqvist2003, loreau2008, loreau2013}\\

Intraspecific trait variation &
Identifies that variation among individuals in a population can form an important component of diversity and affect population dynamics. Individuals could be thought of as ecological assets. &
\citep{bolnick2011}\\

Spatial heterogeneity &
Heterogeneity in habitat may create pockets of diverse ecological assets. &
\citep{oliver2010,parn2012,mccluney2014}\\

%Compensatory dynamics and complementarity & … & [65]\\

%Competitive interactions & … & [37]\\

\midrule
\multicolumn{2}{l}{\textbf{Risk-reduction consequences of diversity}}\\
\midrule

Diversity-stability hypothesis &
Identifies when and why certain types of diversity correspond with certain types of stability. Another name for many of the same concepts that ecological portfolio theory addresses.  &
\citep{ives2007, loreau2013}\\

Statistical averaging &
Some level of reduction in variability in a portfolio is inevitable due to averaging of asset time series. &
\citep{doak1998}\\

Portfolio effect &
A term to represent the benefit of a system existing as a portfolio of ecological assets instead of a single homogeneous asset. &
\citep{tilman1998, schindler2010, thibaut2013, anderson2013}\\

Insurance hypothesis &
Similar to the portfolio effect, but emphasizes that negative covariance reduces the likelihood that all assets will decline at the same time. &
\citep{yachi1999,valone2008}\\

\bottomrule
\label{tab:theory}
\end{longtable}
\end{small}
