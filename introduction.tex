\section{Introduction}\label{introduction}

In the coming century we face a loss of biodiversity on the order of 100--10,000 times greater than average rates in the fossil record \citep{mea2005} --- a rate as fast if not faster than any of the five past mass extinctions \citep{barnosky2011, harnik2012}. Compounding this problem for conservation managers is uncertainty in future climate conditions \citep{heller2009} and the unknown responses of species and communities to those conditions \citep{lavergne2010}. Therefore, urgent questions are: Exactly how big a problem is the loss of biodiversity for the stability of ecological systems? How can conservation biologists communicate the insurance benefit of biodiversity to the public and policy makers? And, how can we apply limited conservation funds to manage biodiversity and limit risk in the face of increasing environmental uncertainty?

Nearly a decade ago, \citet{figge2004} and \citet{koellner2006} laid the foundation for why financial portfolio theory is ideally suited for answering these questions. Financial portfolio theory seems applicable to ecological systems for at least four reasons. First, ecological and financial systems are both structured hierarchically. Groups of populations form metapopulations and groups of species form communities; groups of financial assets form investment funds, which in turn form portfolios. Additionally, ecological and financial managers have similar goals. Ecological resource managers might wish to minimize the probability of population decline while maintaining an acceptable level of hunting of fishing; financial portfolio managers minimize the probability of large economic losses for an acceptable level of expected financial returns \citep{may2008}. Another reason why portfolio theory is ideally suited for ecology is that substantial resources have gone into developing mathematical theory for optimizing financial investments. There is therefore a rich body of theory and experience to draw from. Finally, the portfolio metaphor is an engaging and accessible way for ecologists and conservationists to think about ecological variance and biological diversity and convey the importance of this (often abstract) literature.

A number of recent studies have used financial portfolios as a metaphor, metric, or management approach (Fig.~\ref{fig:traits}) to estimate and communicate the stabilizing benefit of diversity and prioritize its conservation \citep[e.g.][]{schindler2010, ando2011, halpern2011, hoekstra2012, anderson2013, mellin2014}. For example, \citet{moore2010} used the portfolio metaphor to show how increased synchrony of salmon populations could lead to heightened extinction risk. \citet{thibaut2012} used the portfolio-effect metric to quantify the insurance benefit of diversity for reef fish communities. \citet{ando2012} used portfolio optimization to prioritize habitat for conservation that would create wetlands most robust to climate-change uncertainty. Portfolio theory promises to move conservation biology beyond the familiar concepts of the quantity, variety, and distribution of species \citep{mace2005} and into a new dimension that emphasizes elements of variance, covariance, stability, synchrony, and extremeness \citep{loreau2010a, thompson2013}.

But in applying financial theory to conservation biology, where does the portfolio metaphor break down? What exactly are the assets, portfolios, and investors in the financial metaphor? Furthermore, how might one invest in different kinds of ecological portfolios? And how might the differences between financial and ecological data affect our ability to apply financial portfolio theory to conservation biology? Are ecological portfolios just another name for existing ecological theory? Or, can ecological portfolios bring together and provide new insights into a wide variety of ecological theory?

Here, we address these questions by reviewing the emerging literature on ecological portfolios. The purpose of our review is fourfold: review the three recent contrasting applications of ecological portfolios as metaphors, metrics, and management tools; illustrate how ecological portfolios can bring together a wide variety of modern ecological theory; highlight some challenges of applying financial theory to ecological portfolios; and emphasize the utility of portfolio optimization and risk metrics for conservation prioritization.

\bibliographystyle{apalike}\bibliography{/Users/seananderson/Dropbox/tex/jshort,/Users/seananderson/Dropbox/tex/ref3}
