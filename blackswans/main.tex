\chapter[Ecological black swans]{Evidence for black-swan events in animal
  populations\footnotemark[3]}

\footnotetext[3]{T.A. Branch, A.B. Cooper, and N.K. Dulvy are co-authors on this
chapter, which is in preparation for submission to a journal.}
%!TEX root = anderson-etal-blackswan-timeseries.tex

\section{Abstract}

Black swans are statistically improbable events that nonetheless occur---often
with profound consequences. While extremes in the physical environment, such
as monsoons and heat waves, are widely studied and increasing in magnitude and
frequency, it remains unclear the extent to which ecological populations
buffer or suffer from such extremes. Here, we estimate the degree of
heavy-tailedness (presence of black swans) in ecological process noise by
applying a probability model to \NPops~time series from around the world
across \NOrders~taxonomic orders and seven classes. We find strong evidence of
black swans, but they are rare, occurring in
\overallMinPerc--\overallMaxPerc\% of populations: most frequently for birds
(\AvesRangePerc\%) followed by mammals (\MammaliaRangePerc\%) and insects
(\InsectaRangePerc\%). Black swans were predominantly (\percBSDown \%)
downward events and were not explained by any life-history covariates, but
tended to be driven by external perturbations such as climate, severe winters,
predator or parasite cycles, and the combined effects of multiple factors.
Extreme events were more frequently detected for populations with longer time
series and lower levels of process noise; for shorter and noisier time series,
our simulations suggested black-swan dynamics are often not identifiable as
such. The presence of black swans in population dynamics highlights the
importance of developing robust conservation and management
strategies---particularly as the frequency and magnitude of climate extremes
increase over the next century.

\section{Introduction}

Black swans are unexpected extreme events with potentially dramatic
consequences \citep{taleb2007,sornette2009}. One of the most striking black
swans in ecology is the asteroid collision that may have marked the
end-Cretaceous mass extinction 65 million years ago
\citep{alvarez1980,harnik2012}. Today, climate extremes, in concert with
shifts in mean temperature, are expected to cause the greatest ecological and
societal damage \citep{ipcc2012}. But, while extremes in the physical
environment such as wave height, storm severity, and temperature are frequent
events \citep{gaines1993,katz2005}, it remains unclear the extent to which
ecological systems buffer or suffer from black swans \citep{nunez2012}.

There is compelling anecdotal evidence for ecological black swans, but
systematic evidence across taxa has been elusive. A survey of ecologists
indicated that surprising outcomes of field experiments are far more common
than we assume \citep{doak2008}, and events such as the global invasion of
Argentine ants and the mutation of viruses to infect new hosts could be
considered black swans \citep{nunez2012}. In fact, anecdotes of population
catastrophes are numerous and catastrophes may be the most important element
affecting population persistence \citep{mangel1994}. For marine mammal
populations, we have compelling evidence of catastrophes \citep{gerber2001,
  ward2007}, and recently, time-series of marine microbe abundance
\citep{segura2013} and time to extinction for experimental waterflea
populations \citep{drake2014} have been found to follow heavier tailed
distributions than the normal. Despite these examples, as far as we can tell
there have been two systematic surveys for ecological black swans: one with
North American breeding birds and the other with the Global Population
Dynamics Database, which uncovered little clear evidence for them
\citep{keitt1998,allen2001,halley2002}. Nevertheless there are methodological
challenges to their detection \citep{allen2001,ward2007}.

There are two key reasons why we may find little evidence of ecological black
swans. First, they might not exist in higher taxa. Indeed, the majority of
model fitting and risk forecasting assumes that population dynamics are normal
tailed on a log scale \citep[e.g.][]{brook2006a,dennis2006,knape2012}.
Alternatively, black-swan dynamics might exist, but our ability to detect
them requires further development of statistical tools. One such tool is the
generalized extreme value distribution, which has been applied to
environmental data \citep[e.g.][]{katz2005}. This distribution describes the
most extreme event per time interval (e.g.~heaviest rainfall per year), but
requires time series that are sufficiently long to be condensed into time
intervals. Another statistical tool involves fitting a catastrophic mixture
distribution in a state space model to quantify the probability that
population events are extreme, although this is also data intensive
\citep{ward2007}. A third tool is to compare the support for fits of thin- and
heavy-tailed distributions \citep{halley2002}, but this analysis did not
quantify the probability of black swans or allow for population dynamics.

Here, we assess the frequency and magnitude of black-swan dynamics across
\NPops\ populations from a wide array of taxonomic groups---mostly birds,
mammals, insects, and fishes. We then identify characteristics of time series
or intrinsic life-history characteristics that are associated with the
detection of black-swan events and attempt to verify known causes. To
accomplish this, we develop a framework for identifying heavy-tailed process
noise in population dynamics, i.e.\ whether the largest stochastic jumps in
abundance from one time step to the next are more extreme than typically seen
with a log-normal distribution. Our framework allows for a range of population
dynamic models, can incorporate observation uncertainty, and can be easily
applied to abundance time series.

\section{Methods}

To obtain estimates of the probability and magnitude of black-swan events, we
fit population dynamic models to abundance time series from around the world.
For each population, we estimated the shape of the process noise tails by
measuring the degrees of freedom ($\nu$) of a Student t-distribution.

\subsection{Time-series data}

We selected abundance time series from the Global Population Dynamics Database
(GPDD; \citeauthor{gpdd2010} \citeyear{gpdd2010}), which contains nearly 5000
time series of abundance from $\sim$1000 species and $\sim$100 taxonomic
orders. We filtered the data (Supporting materials) to remove populations
from less reliable data sources, and those without sufficient data for our
models, and then interpolated some missing values
\citep[\textit{sensu}][]{brook2006a}. Our interpolation affected only
$\sim$\interpPointsPerc \% of the final data points (Supporting materials~Table~\ref{tab:stats})
and none of the data points that were later considered black-swan events. Our
final dataset contained \NPops~populations across \NOrders~taxonomic orders
and seven taxonomic classes, with a median of \medianTimeSteps~time steps
(range of \minTimeSteps--\maxTimeSteps) (Supporting materials Table~\ref{tab:stats},
Fig.~\ref{fig:all-ts}).

\subsection{Population models}

Our main analysis focuses on the commonly applied Gompertz population dynamics
model \citep[e.g.][]{knape2012,dennis2014,connors2014}. The Gompertz model
represents population growth as a linear function in $\log$ space. If we let
$x_t$ represent the $\log$ abundance ($N$) at time $t$, we can represent the
Gompertz model as:
\begin{align*}
x_t &= \lambda + b x_{t-1} + \epsilon_t\\
\epsilon_t &\sim \mathrm{Student\mhyphen t}(\nu, 0, \sigma).
\end{align*}
The growth parameter $\lambda$ represents the expected growth rate if $N_t =
1$. The model is density independent if $b = 1$, maximally density dependent
if $b = 0$, and inversely density dependent if $b < 0$. Usually, the process
noise $\epsilon_t$ is modelled as normally distributed, but in our paper we
assume it is drawn from a t distribution with scale parameter $\sigma$ and
degrees of freedom $\nu$. In previous analyses of the GPDD, the Gompertz was
most often identified as the most parsimonious population model fit to these
data \citep{brook2006}.

By allowing the process noise to be drawn from a Student-t distribution we can
estimate the degree to which the process deviations have heavy tails and are
thereof evidence of black-swan events (Fig.~\ref{fig:didactic}a, b). For
example, at $\nu = 3$, the probability of drawing a value more than five
standard deviations below the mean is $0.02$, whereas the probability of
drawing such a value from a normal distribution is tiny ($2.9\cdot10^{-7}$).
As the value of $\nu$ approaches infinity, the distribution approaches the
normal distribution (Fig.~\ref{fig:didactic}a, b). While, across populations,
the extreme process deviations might tend to be more frequently upwards or
downwards events, a small random sample from a heavy-tailed distribution can
have many outliers on one side and appear asymmetric \citep{gelman2013a}.
Therefore, fitting a symmetric t distribution is appropriate for the
relatively short population time series in our dataset and is agnostic towards
the detection of upwards or downwards black swans.

\begin{figure}[htbp]
\begin{center}
\includegraphics[width=\textwidth]{blackswans/analysis/t-nu-eg2.pdf}
\caption[An illustration of fitting population dynamic models that allow for heavy
tails, represented by the Student-t degrees of freedom parameter $\nu$.]{
An illustration of fitting population dynamic models that allow for heavy
tails, represented by the Student-t degrees of freedom parameter $\nu$. (a, b)
The probability density for t distributions with a scale parameter of 1 and
different values of $\nu$. Small values of $\nu$ create heavy tails. As $\nu$
approaches infinity the distribution approaches the normal distribution. For
example, at $\nu = 2$, the probability of drawing a value more than five
standard deviations below the mean is $0.02$, whereas the probability of drawing
such a value from a normal distribution is nearly zero ($2.9\cdot10^{-7}$).
(c--e) Simulated population dynamics from a Gompertz model with process noise
drawn from t distributions with three different values of $\nu$. Coloured dots
in panels c and d represent jumps with less than a 1 in 1000 chance of
occurring in a normal distribution. (f--h) Estimates of $\nu$ from models fit
to the times series in panels c--e. Shown are the posterior samples
(histograms), median and interquartile range of the posterior (IQR) (dots and
line segments), and the exponential prior on $\nu$ (dashed lines). Colour
shading behind panels f--h illustrates the region of heavy tails.
}
\label{fig:didactic}
\end{center}
\end{figure}

We fit all models in a Bayesian framework using Stan \citep{stan-manual2014}
via the R computing environment \citep{r2014}. Stan samples from the posterior
distribution with an adaptive version of Hamiltonian Markov chain Monte Carlo
called the No-U-Turn Sampler and generally obtains less correlated samples
than algorithms such as the Gibbs sampler \citep{hoffman2014}. We tested to
ensure that the chains had sufficiently converged and that the sampler had
obtained sufficient independent samples from the posterior ($\widehat{R} <
1.05$, $n_\mathrm{eff} > 200$; Supporting materials).

We chose weakly informative priors to incorporate our understanding of
plausible population dynamics (\citeauthor{gelman2014} \citeyear{gelman2014};
Figs~\ref{fig:didactic}f--h, \ref{fig:priors}). For $\nu$, we chose an
exponential prior with rate parameter of $0.01$ truncated at values above
two---a slightly less informative prior than suggested by
\citet{fernandez1998}. This prior gives only a \basePriorProbHeavy \%
probability that $\nu < 10$ but constrains the sampling sufficiently to avoid
wandering off towards infinity. In any case, for $\nu > 20$ the t distribution
is almost indistinguishable from the normal distribution
(Fig.~\ref{fig:didactic}). Based on the shape of the t distribution, we chose
the probability that $\nu < 10$, Pr($\nu < 10$), to define the probability of
heavy-tailed dynamics. When categorizing a population as heavy or normal tailed,
we used a threshold of 0.5 probability.

We fit alternative population models to test if four key phenomena
systematically changed our conclusions. Autocorrelation has been suggested as
a reason for increased observed variability of abundance time series through
time, which could create apparent heavy tails \citep{inchausti2002};
therefore, we fit a model that included serial correlation in the residuals.
Additionally, previous work has modelled abundance or growth rates without
accounting for density dependence \citep{halley2002,segura2013}; therefore, we
fit a simpler model in which we assumed density independence. Third,
observation error could bias parameter estimates \citep{knape2012} or mask our
ability to detect heavy tails \citep{ward2007}; therefore, we fit a model
where we allowed for a fixed quantity of observation error ($0.2$ standard
deviations on a log scale). Finally, the Gompertz model assumes that
population growth rate declines linearly with log abundance. Therefore, we
also fit an alternative model, the Ricker-logistic model, which assumes that
population growth rate declines linearly with abundance itself (Supporting
materials).

In addition to alternative population models, we investigated the sensitivity
of our results to weaker and stronger priors (exponential rate parameter $=
0.005, 0.02$; Supporting materials Fig.~\ref{fig:priors}, Supporting materials). We used
simulated data to test how easily we could detect $\nu$ given different sample
sizes and to ensure we could recover unbiased parameter estimates from the
Gompertz model (Supporting materials).


\subsection{Covariates of population dynamic black swans}

We investigated possible covariates of heavy-tailed population dynamics
visually and through multilevel modelling. We plotted characteristics of the
time series ($\sigma$, $\lambda$, $b$, and time-series length) along with two
life-history characteristics (body length and maximum lifespan obtained from
\citet{brook2006a}) against our estimated probability of heavy tails, Pr$(\nu <
10)$. We formally investigated these relationships by fitting beta
regression multilevel models \citep{ferrari2004}. The beta probability
distribution can represent continuous response data that range between zero
and one; we fit our models with a logit link as is common for beta and logistic
regression \citep{ferrari2004}. We incorporated standard deviations around the
means for covariates that were derived from Gompertz model parameter
estimates. To account for broad patterns of phylogenetic relatedness, we
allowed for hierarchical intercepts at the taxonomic class, order, and species
level (Supporting materials). We fit our model in Stan with weakly
informative priors on the coefficients \citep{gelman2008d} and variance
parameters (\citeauthor{gelman2006c} \citeyear{gelman2006c};
\citeauthor{gelman2014} \citeyear{gelman2014}; Supporting materials).

Finally, we investigated a sample of populations that our method categorized as
having a high probability of heavy tails (Pr$(\nu < 10) > 0.5$). Where
possible, we found the documented causes of ecological black swans in the
primary data source cited in the GPDD or in other literature describing the
population.

\section{Results}

We found strong, but rare, evidence for black-swan population dynamics. By
defining black-swan dynamics as a greater than $0.5$ probability that $\nu <
10$, our main Gompertz model found evidence for heavy tails most frequently
for birds (\birdPH\%) followed by mammals (\mammalsPH\%), and insects
(\insectsPH\%) (Fig.~\ref{fig:nu-coefs}, Supporting materials Table~\ref{tab:causes-supp}). Black
swans were taxonomically widespread, occurring in \POrdersHeavy\% of taxonomic
orders. Accounting for time series length and partially pooling inference
across taxonomic class and order with a multilevel model (Supporting
materials), there was stronger evidence for black swans in insect
populations than is visually apparent in Fig.~\ref{fig:nu-coefs}---four of 10
orders with the highest median probability of heavy tails were insect
orders---however, there was considerable uncertainty in these estimates
(Fig.~\ref{fig:posteriors}a).

\begin{figure}[htbp]
\begin{center}
\includegraphics[width=0.44\textwidth]{blackswans/analysis/nu-coefs-2.pdf}

\caption[Estimates of population dynamics heavy-tailedness for \nuCoefPopN\
  populations of birds, mammals, insects, and fishes.]{Estimates of population dynamics heavy-tailedness for \nuCoefPopN\
  populations of birds, mammals, insects, and fishes. Small values of $\nu$
  approximately ($< 10$) suggest heavy-tailed black-swan dynamics; larger values of
  $\nu$ suggest approximately normal-tailed dynamics. Vertical points and line
  segments represent posterior medians and 50\% / 90\% credible intervals for
  individual populations. Inset plots show probability that $\nu < 10$
  (probability of heavy tails) for populations arranged by taxonomic order and
  sorted by decreasing mean Pr($\nu < 10$). Taxonomic orders with three or
  fewer populations in panel a are omitted for space. Red to yellow points
  highlight populations with a high to moderately high probability of
  heavy-tailed black-swan dynamics.}

\label{fig:nu-coefs}
\end{center}
\end{figure}

\begin{figure}[htbp]
\begin{center}
\includegraphics[width=0.45\textwidth]{blackswans/analysis/order-posteriors-covariates.pdf}

\caption[Posterior probability distributions from beta regression multilevel
  models.]{Posterior probability distributions from beta regression multilevel
  models. (a) Taxonomic-order-level posterior densities of Pr($\nu < 10$)
  (approximately the probability of heavy tails) after accounting for
  time-series length. Estimates are at the geometric mean of time series
  length across all the data (approximately 27 time steps). Colour shading
  refers to taxonomic class (yellow: fishes, green: insects, purple: birds,
  and red: mammals). Dotted vertical line in panel a indicates the
  Pr($\nu < 10$) from the prior distribution. (b) Main effect
  posterior densities for potential covariates of Pr($\nu < 10$). The beta
  regression models were fit on a logit scale with hierarchical intercepts for
  taxonomic class, taxonomic order, and species. All covariates were
  standardized by subtracting their mean and dividing by twice their standard
  deviation. In both panels, short vertical line segments within the density
  polygons indicate median posterior estimates.}

\label{fig:posteriors}
\end{center}
\end{figure}


The majority of our heavy-tailed estimates were robust to alternative
population models, observation error, and choice of priors. Our conclusions
were not systematically altered when we included an autocorrelation structure
in the residuals, modelled population growth rates without density dependence,
or modelled the population dynamics as Ricker-logistic (Supporting materials Fig.~\ref{fig:alt}).
However, setting observation error standard deviation to $0.2$ increased the
median estimate of $\nu$ from $<10$ to $\ge 10$ in $\baseNuTenObsTenSwitch$ of
$\baseNuTen$ populations, although the majority of $\nu$ estimates remained
qualitatively similar (Supporting materials Fig.~\ref{fig:alt}). The strength of the prior on $\nu$
had little influence on estimates of black-swan dynamics
(Supporting materials~Fig.~\ref{fig:alt-priors}). Our simulation testing shows that, if anything,
our models underpredict the true magnitude and probability of heavy tailed
events---especially given the length of the time series in the GPDD
(Supporting materials Figs~\ref{fig:sim-nu}, \ref{fig:sim-prob}).

Across populations, the probability of observing black-swan dynamics was
positively related to time-series length and negatively related to magnitude
of process noise ($\sigma$) but not clearly related to population growth rate
($\lambda$), density dependence ($b$), or maximum lifespan
(Figs~\ref{fig:posteriors}b, \ref{fig:correlates}). Longer time-series length
was the strongest covariate of observing black-swan dynamics. For instance,
the expected probability density below $\nu = 10$ was about \pIncHeavyNThirtyNSixty~times
greater for a population with 60 time steps compared
to one with 30 time steps. However, the absolute change in probability with
increased time series length was small ($\pHeavyNSixty$ vs.\ $\pHeavyNThirty$
in the previous example, Fig.~\ref{fig:correlates}).

\begin{figure}[htbp]
\begin{center}
\includegraphics[width=0.9\textwidth]{blackswans/analysis/correlates-p10.pdf}

\caption[Potential covariates of heavy-tailed population dynamics.]{Potential covariates of heavy-tailed population dynamics (indicated
  by a high probability that $\nu < 10$). Shown are (a--c) parameters from the
  Gompertz heavy-tailed population model ($\sigma$, $\lambda$, $b$), (b)
  number of time steps, (c) body length, and (d) lifespan. For the Gompertz
  parameters, $\sigma$ refers to the scale parameter of the Student-t
  process-noise distribution, $\lambda$ refers to the expected log abundance
  at the next time step at an abundance of one, $b$ refers to the density
  dependence parameter ($1$ is maximally density independent, $0$ is maximally
  density dependent, and $<0$ is inversely density dependent). Circles
  representing a few sharks, crustaceans, and gastropods are filled in white.
  Median and 90\% credible interval posterior predictions of a beta regression
  multilevel model are shown in panels a and d where there was a high
  probability the slope coefficient was different from zero
  (Fig.~\ref{fig:posteriors}b).}

\label{fig:correlates}
\end{center}
\end{figure}



\LTcapwidth=\textwidth
\bibpunct{}{}{;}{a}{}{;}

\begin{small}
\begin{longtable}{>{\RaggedRight}m{2.0cm}>{\RaggedRight}p{3.0cm}>{\RaggedRight}p{7.0cm}>{\RaggedRight}p{2.0cm}}

\caption[Example population dynamic black swans from the Global Population
  Dynamics Database and a description of their causes.]{Example population dynamic black swans from the Global Population
  Dynamics Database and a description of their causes. Red and blue dots
  indicate downward and upward events that have a $10 \cdot 10^{-4}$ probability or less of
  occurring if the population dynamics were explained by a Gompertz model
  with normally distributed process noise. These populations are
  a sample from the heavy-tailed populations we could verify
  (Table~\ref{tab:causes-supp}).}\\

\toprule
Time series (log scale) & Population & Black swan description & Reference \\
\midrule

\includegraphics[width=2cm]{blackswans/analysis/sparks/6528} &
Shag,
\textit{Phalacrocorax aristotelis},
UK &
Shortage of nest sites reduced productivity; red-tide event in 1968 caused
extreme mortality; no longer a nest shortage; population rapidly increased &
\citep{potts1980}\\

\includegraphics[width=2cm]{blackswans/analysis/sparks/10007} &
Water vole,
\textit{Arvicola terrestris},
UK &
Short-term population cycles from predator interactions combined with long-term
environmental cycle caused sharp downswing  &
\citep{saucy1994}\\

\includegraphics[width=2cm]{blackswans/analysis/sparks/7115} &
Fur seal,
\textit{Arctocephalus pusillus},
South Africa &
Strong decreases in harvesting, loss of predators, and diamond mining
regulations reducing human traffic caused sharp upswings  &
\citep{shaughnessy1982}\\

\includegraphics[width=2cm]{blackswans/analysis/sparks/10113} &
Willow grouse,
\textit{Lagopus lagopus},
UK &
Parasite and predation effects interacted to cause low years  &
\citep{dobson1995}\\

\includegraphics[width=2cm]{blackswans/analysis/sparks/10162} &
Red grouse,
\textit{Lagopus lagopus scoticus},
UK &
Good environmental conditions produced high numbers and vulnerable populations;
bad conditions and overcrowding combined to create crashes  &
\citep{mackenzie1952}\\

\includegraphics[width=2cm]{blackswans/analysis/sparks/1235} &
Wren,
\textit{Troglodytes troglodytes},
UK &
Severe winters where food was buried under snow caused population crash &
\citep{newton1998} \\

\includegraphics[width=2cm]{blackswans/analysis/sparks/20579} &
Grey heron,
\textit{Ardea cinerea},
UK &
Severe winters in 1929, 1940--1942, and 1962--1963; 1963 event so severe that
recovery took three times longer than expected &
\citep{stafford1971} \\

%\includegraphics[width=2cm]{blackswans/analysis/sparks/20580} &
%Chamois, \textit{Rupicapra rupicapra}, Switzerland &
% &
%\citep{brook2006a}\\

%\includegraphics[width=2cm]{blackswans/analysis/sparks/5019} &
%Barbary macaque,  &
% &
%REF\\

%\includegraphics[width=2cm]{blackswans/analysis/sparks/9675} &
%Carrot fly (\textit{Psila rosae}, Finland &
% &
%\citep{markkula1965}\\

%\includegraphics[width=2cm]{blackswans/analysis/sparks/10139} &
%Grey heron (\textit{Ardea cinerea}), UK &
% &
%\citep{stafford1971}\\

\bottomrule
\label{tab:sparks}
\end{longtable}
\end{small}

% reset citation style:
\bibpunct{(}{)}{;}{a}{}{;}


We examined all time series with published explanations of why the black-swan
events occurred (Table~\ref{tab:sparks} and Supporting materials~Table~\ref{tab:causes-supp}). The
majority of documented events (\percBSDown \%) were downward black swans and
involved a combination of multiple factors. For example, a synchronization of
environmental- and predation-mediated population cycles are thought to have
caused a downward black-swan event for a water vole (\textit{Arvicola
  terrestris}) population \citep{saucy1994}. Other black swans were the result
of a sequence of extreme climate events on their own. For instance, severe
winters in 1929, 1940--1942, and 1962--1963 were associated with black-swan
downswings in grey heron (\textit{Ardea cinerea}) abundance in the United
Kingdom \citep{stafford1971}. Our analysis finds that the last event was a
combination of two black-swan events in a row and it took the population three
times longer to recover than predicted \citep{stafford1971}. Downwards black
swans were sometimes followed by upwards black swans. For example, during a
period of population crowding and nest shortages, a population of European
shag cormorants (\textit{Phalacrocorax aristotelis}) on the Farne Islands,
United Kingdom, declined suddenly following a red tide event in 1968
\citep{potts1980}. This freed quality nest sites for first-time breeders,
productivity rapidly increased, and the population experienced a rapid upswing
in abundance \citep{potts1980}.


\section{Discussion}

We found strong evidence for black swans (heavy-tailed process noise) in
\overallMinPerc--\overallMaxPerc\% of ecological time series. Black swans were
usually (\percBSDown \%) downward events and were detected more frequently in
longer time series and in populations with a smaller magnitude of process
noise. Black swans were not associated with density dependence, population
growth rate, or lifespan. In verified cases, black-swan events were often a
result of the combination of extreme climate, predators and parasite cycles,
and strong changes in human pressures. Our empirical results, sensitivity
analyses, and simulation tests suggest that estimating the tail shape of
process noise is a viable method of detecting black-swan population dynamics
and if anything will underestimate the probability of black-swan events. The
presence of black swans highlights the importance of developing management
strategies that detect quickly, respond to, and are robust to extremes in
population dynamics---particularly as the frequency and magnitude of climatic
extremes increase over the next century \citep{easterling2000,ipcc2012}.

Our results clarify previous related analyses. An analysis with an older
version of the GPDD assessed the distribution of abundance time series but
focused on identifying if the log-normal distribution was the most frequently
parsimonious model \citep{halley2002}. For heavy-tailed distributions,
\citet{halley2002} fit the extremely heavy-tailed Cauchy distribution and the
four-parameter Levy stable distribution and found little information criteria
support for these distributions in longer time series. However, black-swan
events are by definition rare, and the majority of time series in the GPDD are
short for these purposes. Therefore, we would not expect to observe black
swans in a large proportion of populations. By quantifying the probability of
heavy-tails and allowing for non-stationary time series and density
dependence, our analysis allows for a more nuanced description of the evidence
of ecological black swans. In an earlier study, \citet{keitt1998} described
heavy (power law) tails in breeding bird population abundance. However, this
finding was challenged by \citet{allen2001}, who showed that mixing data
across species could falsely generate heavy tails.
%This rebuttal supports our
%use of the t distribution to represent heavy tails, since a t distribution can
%be represented by a mixture of normal distributions \citep[with the same mean
%and inverse-gamma-distributed variances,][]{gelman2014}.

We might expect to observe black-swan dynamics in ecological time series
because of unmodelled intrinsic properties of populations or extrinsic forces
acting on populations. Since a t-distribution can be formed by a mixture of
normal distributions \citep[with the same mean and inverse-gamma-distributed
variances,][]{gelman2014}, we could observe heavy tails if we miss some
underlying mixture of intrinsic processes \citep{allen2001}. That process
might be an aggregation of populations across space, or population diversity,
or an intrinsic change in population variability through time. Extrinsic
forces could also cause black-swan dynamics \citep[e.g.][]{nunez2012}. These
forces could be extreme themselves. For example, extreme climate, predation
from (or competition with) other species experiencing black swans, or sharp
changes in human pressure such hunting, fishing, or habitat destruction might
cause black swans. Alternatively, the synchrony of multiple ``normal''
extrinsic forces could give rise to black-swan ecological dynamics. This could
occur with a synergistic interaction \citep[e.g.][]{kirby2009} or even if
non-synergistic forces experience a rare alignment \citep{denny2009}.

There are a number of caveats when considering the generality of our results.
The GPDD data represent a taxonomically and geographically biased sample of
populations---the longer time series we focus on are dominated by commercially
and recreationally important species and a disproportionate number of
populations are located in the United Kingdom. Although we would expect to
find qualitatively similar evidence for black swans in many other large
taxonomic or geographic samples of populations, the common forces driving
those black swans would likely differ (e.g.~severe winters in
Table~\ref{tab:sparks}). In addition to a possibly biased sample of
populations, some black-swan detections could just be recording mistakes, or
conversely, some extreme observations may have been discarded or altered if
they were erroneously suspected of being recording mistakes. Indeed, we
discarded three of the populations that our method initially identified as
heavy tailed because they turned out to be data-entry errors (Supporting
materials). A final caveat is that the temporal scales of observation and
population dynamics vary considerably across populations in the GPDD and this
likely influences the detection of heavy tails. As an example, if we make
frequent observations relative to generation time (e.g.~for many large-bodied
mammals) we will average across generations and perhaps miss black swans.
Conversely, if we census populations infrequently relative to generation time
(e.g.~many insects in the GPDD) the recorded data may average across extreme
and less-extreme events and also dampen black-swan dynamics.

Recognizing the prevalence of heavy-tailed dynamics suggests a number of
policy directions. Ecological resource management can draw from other
disciplines that focus on heavy tails. For example, earthquake preparedness
and response is focussed on black-swan events. Similarly to ecological black
swans, we can rarely predict the specific location and timing of large
earthquakes. But, earthquake preparedness involves spatial planning based on
forecast probabilities to focus early detection efforts and develop disaster
response plans. A similar focus might benefit resource management once our
ability to predict the spatial probability and covariates of ecological black
swans improves. The presence of black swans also suggests that we develop
management policy that is robust to heavy tails. For instance, setting target
population abundances that are appropriately set back from critical limits may
buffer black-swan events \citep[e.g.][]{caddy1996}, and maintaining genetic,
phenotypic, and behavioural diversity may allow some components of populations
to persist when others are affected by disease or extreme environmental forces
\citep[e.g.][]{hilborn2003, schindler2010, anderson2014}. Finally, extreme and
unexpected, surprising, or counterintuitive ecological dynamics offer a
tremendous opportunity to learn about ecological systems, evaluate when our
models break down, and adjust future management policy \citep{doak2008,
  pine-iii2009, lindenmayer2010}.

Our results suggest a number of research questions related to ecological black
swans. Given that black swans do occur, can we forecast the probability of
black swans in space and time? Furthermore, what management policies allow us
to detect them quickly after they happen? Can we isolate the components of
ecological dynamics that experience black swans by moving from
phenomenological models such as the Gompertz to mechanistic models that, for
example, take into account recruitment dynamics? We expect that greater
insight into the mechanisms and covariates of ecological black swans may be
best obtained through specific geographic and taxonomic subsets of data where
longer time series with low levels of observation error are available
\citep[e.g.][]{segura2013}.

Most importantly, what is the impact of allowing for black swans in forecasts
of ecological risk? Even extremely rare catastrophes can have a profound
influence on population persistence \citep{mangel1994}. In recent decades,
ecology has moved toward focussing on aspects of variance in addition to mean
responses \citep[e.g.][]{loreau2010a, thompson2013}. Our results suggest that
an added focus on ecological extremes represents the next frontier,
particularly in the face of increased climate extremes \citep{meehl2004,
  ipcc2012, thompson2013}. Financial analysts are concerned with the shape of
the downward tails of financial returns because these directly impact
estimates of risk---the probability of a specific magnitude of undesired event
occurring \citep{rachev2008}. A comparable focus in ecology would increase our
estimates of extinction risk, since these would be disproportionately impacted
by downward black-swan events.

\section{Acknowledgements}

We thank J.W. Moore, J.D. Yeakel, and other members of the Earth to Ocean
research group for helpful discussions and comments. We are grateful to the
contributors and maintainers of the Global Population Dynamics Database and to
Compute Canada's WestGrid high-performance computing resources. Silhouette
images were obtained from \texttt{phylopic.org} under Creative Commons
licenses; sources are listed in the Supporting materials. Funding was provided
by a Simon Fraser University Graduate Fellowship (SCA), the Natural Sciences
and Engineering Research Council of Canada (NKD, ABC), the Canada Research
Chairs Program (NKD).

% \section{Supporting Information}
%
% The following supporting information is available online for this article:\\
% Tables S1 and S2\\
% Figures S1--S6\\
% Example Stan code for a heavy-tailed Gompertz model and multilevel beta
% regression\\
% The GPDD IDs used in our analysis\\
% All code and data to recreate our analysis are available at:\\
% \url{https://github.com/seananderson/heavy-tails}
%
% \renewcommand{\baselinestretch}{\tighttextstretch}
% \normalsize
% \bibliographystyle{apalike}
% \bibliography{/Users/seananderson/Dropbox/tex/jshort,/Users/seananderson/Dropbox/tex/ref3}
%
% \clearpage
% \renewcommand{\baselinestretch}{\textstretch}
% \normalsize

%\section{Tables}


%\clearpage


%\clearpage

%\clearpage


%\clearpage
\chapter[Supporting materials]{Supporting materials: Evidence for black-swan
events in animal populations}
%!TEX root = anderson-etal-blackswan-timeseries.tex

\begin{centering}
\LARGE
%\[1.0em]
\end{centering}

\section{Data selection}

We applied the following data selection and quality-control rules to the
Global Population Dynamics Database (GPDD):

\begin{enumerate}

\item To remove populations with unreliable population indices that could be
  strongly confounded with economics and sampling effort, we removed all
  populations with a sampling protocol listed as \texttt{harvest} as well
  populations with the words \texttt{harvest} or \texttt{fur} in the cited
  reference title.

\item We removed all populations with uneven sampling intervals, i.e.\ we
  removed populations that did not have a constant difference between the
  ``decimal year begin'' and ``decimal year end'' columns.

\item We removed all populations rated as $< 2$ in the GPDD quality assessment
  (on a scale of $1$ to $5$, with $1$ being the lowest quality data)
  \citep[following][]{sibly2005, ziebarth2010}.

\item Populations with negative abundance values were removed. Of the
  populations that remained at the end of our other filtering rules, the
  remaining populations with negative abundances listed were all from time
  series that had been standardized by subtracting the mean and dividing by the
  standard deviation. We verified this by locating the original papers the
  datasets were extracted from: \citet{colebrook1978} for zooplankton and
  \citet{lindstrom1995} for grouse. Since the papers did not include the
  original mean time-series values we could not back transform these data
  points.

\item We filled in all missing time steps with \texttt{NA} values and imputed
  single missing values with the geometric mean of the previous and following
  values. We chose a geometric mean to be linear on the log scale that the
  Gompertz and Ricker-logistic models were fit on.

\item We filled in single recorded values of zero with the lowest non-zero
  value in the time series \citep[following][]{brook2006a}. This assumes that
  single values of zero result from abundance being low enough that censusing
  overlooked individuals that were actually present. We turned multiple zero
  values in a row into \texttt{NA} values. This implies that multiple zero
  values were either censusing errors or caused by emigration. Regardless, our
  population models were fit on a multiplicative (log) scale and so could not
  account for zero abundance. To avoid distorting the original data too
  strongly, we removed populations in which we filled in more than four zeros.

\item We removed all populations without at least four unique values
  \citep[following][]{brook2006a}.

\item We removed all populations with four or more identical values in a row
  since these suggest either recording error or extrapolation between two
  observations.

\item We then wrote an algorithm to find the longest unbroken window of
  abundance (no \texttt{NA}s) with at least $20$ time steps in each population
  time series. If there were any populations with multiple windows of identical
  length, we took the most recent window. This is a longer window than used in
  some previous analyses \citep[e.g.][]{brook2006a}, but since our model
  attempts to capture the shape of the distribution tails, our model requires
  more data.

\item We removed GPDD Main IDs \texttt{20531} and \texttt{10139}, which we
  noticed were duplicates of \texttt{20579} (a heron population).
  \texttt{20579} contained additional years of data not present in
  \texttt{10139}. We removed a limited number of populations from class
  Angiospermopsida and Bacillariophyceae to focus the taxonomy in our analysis
  on animals. We also removed any populations with an \texttt{Unknown}
  taxonomic class.

\item Finally, we removed populations with the following GPDD Main IDs, which
  we discovered were data entry errors when verifying the populations with
  suspected black swans: \texttt{1207} because the 1957 data point was entered
  as 2 but should have been 27 \citep{kendeigh1982}, \texttt{6531} because the
  1978 data point was entered as 7 but should have been 47 \citep{minot1986},
  and \texttt{6566} because some of the data did not match the graph
  \citep{heessen1996}.

\end{enumerate}

\noindent
We provide a supplemental figure of all the time series included in our
analysis and indicate which values were interpolated (non-zero interpolations)
(\percImputedPops\% of populations had at least one point interpolated but
only \percImputedPoints\% of the total observations were interpolated)
(Fig.~\ref{fig:all-ts}). Note that interpolation is highly unlikely to lead to
black-swan detections, since black swans involve extreme increases or
decreases. Table~\ref{tab:stats} shows the final taxonomic breakdown and the
number of populations with interpolated values.

\section{Details on the heavy-tailed Gompertz probability model}

For the Gompertz model, our weakly-informative priors (Fig.~\ref{fig:priors})
were:
\begin{align*}
b &\sim \mathrm{Uniform}(-1, 2)\\ \lambda &\sim \mathrm{Normal}(0, 10^2)\\
\sigma &\sim \mathrm{Half\mhyphen Cauchy} (0, 2.5)\\ \nu &\sim
\mathrm{Truncated\mhyphen Exponential}(0.01, \mathrm{min.} = 2).
\end{align*}
Our prior on $b$ was uninformative between values of $-1$ and $2$. We would
not expect values of $b$ with levels of density dependence as low as $-1$
(very strong inverse density dependence), nor would we generally expect values
above $1$. We allowed values of $b$ above $1$ to allow for non-stationary time
series of growth rates. The estimates of $b$ were well within these bounds.
Our prior on $\lambda$ was very weakly informative within the range of
expected values for population growth and is similar to the default priors
suggested by \citet{gelman2008d} for intercepts of regression models. Our
Half-Cauchy prior on $\sigma$ follows \citet{gelman2006c} and
\citet{gelman2008d} and the specific scale parameter of $2.5$ is based on our
expected range of the value in nature from previous studies
\citep[e.g.][]{connors2014}. In our testing of a subsample of populations, our
parameter estimates were not qualitatively changed by switching to an
uninformative uniform prior on $\sigma$, but the weakly informative
Half-Cauchy prior substantially sped up chain convergence.

Our prior on $\nu$ was based on \citet{fernandez1998}. They chose an
exponential rate parameter of $0.1$. We chose a less informative rate
parameter of $0.01$ and truncated the distribution at $2$, since at $\nu < 2$
the variance of the t distribution is undefined. This prior gives only a
$7.7$\% probability that $\nu < 10$ but constrains the sampling sufficiently
to avoid wandering off towards infinity---above approximately $\nu = 20$ the t
distribution is so similar to the normal distribution
(Fig.~\ref{fig:didactic}) that time series of the length considered here are
unlikely to be informative about the precise value of $\nu$. In the scenario
where the data are uninformative about heavy tails
(e.g.~Fig.~\ref{fig:didactic}e,~h), the posterior will approximately match the
prior (prior median $= 71$, mean $= 102$) and the metrics used in our paper
(e.g.~Pr$(\nu < 10) > 0.5$) are unlikely to flag the population as heavy
tailed.

We fit our models with Stan 2.4.0 \citep{stan-manual2014}, and R 3.1.1
\citep{r2014}. We began with four chains and $2000$ iterations, discarding the
first $1000$ as warm up (i.e.~4000 total samples). If $\hat{R}$ (the potential
scale reduction factor---a measure of chain convergence) was greater than
$1.05$ for any parameter or the minimum effective sample size,
$n_\mathrm{eff}$, (a measure of the effective number of uncorrelated samples)
for any parameter was less than $200$, we doubled both the total iterations
and warm up period and sampled from the model again. These thresholds are in
excess of the minimums recommended by \citet{gelman2006a} of $\hat{R} < 1.1$
and effective sample size $> 100$ for reliable point estimates and confidence
intervals. In the majority of cases our minimum thresholds were greatly
exceeded. We continued this procedure up to $8000$ iterations ($16000$ total
samples) by which all chains were deemed to have sufficiently converged. These
chain lengths may seem low to those familiar with software such as WinBUGS or
JAGS, but the No-U-Turn Hamiltonian Markov chain Monte Carlo Sampler in Stan
generally requires far fewer iterations to obtain equivalent effective sample
sizes \citep{stan-manual2014}.

\section{Alternative priors}

To test if the prior on $\nu$ influenced our estimate of black-swan dynamics,
we refit our models with weaker and stronger priors. Our base model used a
prior on $\nu$ of Truncated-Exponential(0.02, min.\ = 2). For a weaker prior
we used Truncated-Exponential(0.005, min.\ = 2) and for a stronger prior we
used Truncated-Exponential(0.02, min.\ = 2) (Fig.~\ref{fig:priors}). Note that
the base and weaker priors are relatively flat within the region of $\nu <
20$, which is the region we are mostly concerned about when categorizing populations
as heavy- or thin-tailed.

Our results show that these weaker and stronger priors would have little
influence on our conclusions about heavy-tailed dynamics
(Fig.~\ref{fig:alt-priors}). When the data are informative about tail
behaviour (i.e.\ when there is strong evidence of low $\nu$ values,
upper-right of Fig.~\ref{fig:alt-priors}), the prior has little impact on the
estimate of $\nu$. When the data are less informative about $\nu$ (i.e.\ when
there are no or few tail events and time series are short or noisy), the prior
can pull the estimate of $\nu$ towards larger or smaller values
(Fig.~\ref{fig:alt-priors}). The vast majority of the populations with Pr$(\nu
< 10)$ in the base prior were not altered qualitatively by this range of prior
strength.

\section{Alternative population models}

We fit four alternative population models to the time-series data to check how
they would influence our conclusions. Our alternative models allowed for
autocorrelation in the residuals, assumed no density dependence, allowed for
observation error, or assumed a Ricker-logistic functional form. The range of
percentages of black swans by taxonomic class cited in the abstract and
results are based on lower and upper limits across our main Gompertz model and
these four alternative models.

\subsection{Autocorrelated residuals}

We considered a version of the Gompertz model in which an autoregressive
parameter was fit to the process noise residuals:
\begin{align*}
x_t &= \lambda + b x_{t-1} + \epsilon_t\\
\epsilon_t &\sim \mathrm{Student\mhyphen t}(\nu, \phi \epsilon_{t-1}, \sigma).
\end{align*}
In addition to the parameters in the original Gompertz model, this model
estimates an additional parameter $\phi$, which represents the correlation of
subsequent residuals. Based on the results of previous analyses with the GPDD
\citep[e.g.][]{connors2014} and the chosen priors in previous analyses
\citep[e.g.][]{thorson2014a} and to greatly speed up chain convergence when
running our model across all populations, we placed a weakly informative prior
on $\phi$ that assumed the greatest probability density near zero with the
reduced possibility of $\phi$ being near $-1$ or $1$. Specifically, we chose
$\phi \sim \mathrm{Truncated\mhyphen Normal}(0, 1, \mathrm{min.} = -1,
\mathrm{max.} = 1)$. The MCMC chains did not converge for
\modelsNoConvergeAROne\ populations according to our criteria ($\widehat{R} <
1.05, n_\mathrm{eff} > 200$) after 8000 iterations of four chains. This
  included only \modelsNoConvergeAROneHeavyBase\ populations in which Pr($\nu
  < 10$) $> 0.5$ categorized them as heavy in the main Gompertz model. We did
  not include these models in Fig.~\ref{fig:alt}.

\subsection{Assumed density independence}\label{assumed-density-independence}

We fit a simplified version of the Gompertz model in which the density
dependence parameter $b$ was fixed at $1$ (density independent). This is
equivalent to fitting a random walk model (with drift) to the $\log$
abundances or assuming the growth rates are drawn from a stationary
distribution. The model was as follows:
\begin{align*}
x_t &= \lambda + x_{t-1} + \epsilon_t\\
\epsilon &\sim \mathrm{Student\mhyphen t}(\nu, 0, \sigma).
\end{align*}
We fit this model for three reasons: (1) it is computationally simpler and so
provides a check that our more complicated full Gompertz model was obtaining
reasonable estimates of $\nu$, (2) it provides a test of whether density
dependence was systematically affecting our perception of heavy tails, (3) it
matches how some previous authors have modelled heavy tails without accounting
for density dependence \citep{segura2013}.

\subsection{Assumed observation error}

Observation error can bias parameter estimates \citep[e.g.][]{knape2012} and
is known to affect the ability to detect extreme events \citep{ward2007}. In
our main analysis, we fit a model that ignored observation error. One way to
account for observation error would be to fit a full state-space model that
simultaneously estimates the magnitude of process noise and observation error.
However, simultaneously estimating observation and process noise is a
challenging problem (e.g.\ because the observation and process noise
parameters tend to negatively covary in model fitting) and is known to
sometimes result in identifiability issues with the Gompertz population model
\citep{knape2008}. Furthermore, our model was applied to hundreds of time
series, often of short length (as few as 20 time steps) and our model
estimates an additional parameter---the shape of the process deviation
tails---potentially making identifiability and computational issues even
greater. Therefore, we considered a version of the base Gompertz model that
allowed for a fixed level of observation error:
\begin{align*}
U_t &= \lambda + b U_{t-1} + \epsilon_t\\
x_t &\sim \mathrm{Normal}(U_t, \sigma_\mathrm{obs}^2)\\
\epsilon_t &\sim \mathrm{Student\mhyphen t}(\nu, 0, \sigma_\mathrm{proc}),
\end{align*}
where $U$ represents the unobserved state vector, $\sigma_\mathrm{obs}$
represents the standard deviation of observation error (on a log scale), and
$\sigma_\mathrm{proc}$ represent the process noise scale parameter. We set
$\sigma_\mathrm{obs}$ to $0.2$, which represents the upper limit of values
often used in simulation analyses \citep[e.g.][]{valpine2002, thorson2014b}.

\subsection{Ricker-logistic}

We also fit a Ricker-logistic model:
\begin{align*}
x_t &= x_{t-1} + r_{\mathrm{max}}\left(1 - \frac{N_{t-1}}{K}\right) + \epsilon_t\\
\epsilon_t &\sim \mathrm{Student\mhyphen t}(\nu, 0, \sigma),
\end{align*}
where $r_\mathrm{max}$ represents the theoretical maximum population growth
rate that is obtained when $N_t$ (abundance at time $t$) $= 0$. The parameter
$K$ represents the carrying capacity and, as before, $x_t$ represents the
$\log$ transformed abundance at time $t$. The Ricker-logistic model assumes a
linear decrease in population growth rate with increases in abundance. In
contrast, the Gompertz model assumes a linear decrease in population growth
rate with increases in \textit{log} abundance ($x_t$)
\citep[e.g.][]{thibaut2012}.

To fit the Ricker-logistic models, we chose a prior on $K$ uniform between
zero and twice the maximum observed abundance (\citet{clark2010} chose uniform
between zero and maximum observed, which is more informative). We set the
prior on $r_\mathrm{max}$ as uniform between 0 and 20 as in \citet{clark2010}.
We used the same priors on $\nu$ and $\sigma$ as in the Gompertz model.

\section{Simulation testing the model}

We performed two types of simulation testing. First, we tested how easily the
Student-t distribution $\nu$ parameter could be recovered given different true
values of $\nu$ and different sample sizes. Second, we tested the ability of
the heavy-tailed Gompertz model to obtain unbiased parameter estimates of
$\nu$ given that a set of process deviations was provided in which the
effective $\nu$ value was close to the true $\nu$ value.

We separated our simulation into these two components to avoid confounding two
issues. (1) With smaller sample sizes, there may not be a stochastic draw from
the tails of a distribution. In that case, no model, no matter how perfect,
will be able to detect the shape of the tails. (2) Complex models may return
biased parameter estimates if there are conceptual, computational, or coding
errors. Our first simulation tested the first issue and our second simulation
tested the latter. In general, our simulations show that, if anything, our
model under predicts the magnitude and probability of heavy tailed
events---especially given the length of the time series in the GPDD.

\subsection{Estimating $\nu$ from a stationary t distribution}

First, we tested the ability to estimate $\nu$ given different true values of
$\nu$ and different sample sizes. We took stochastic draws from t
distributions with different $\nu$ values ($\nu = 3, 5, 10,$ and $10^6$
[$\approx$ normal]), with central tendency parameters of $0$, and scale
parameters of $1$. We started with $1600$ stochastic draws and then fit the
models again at the first $800, 400, 200, 100, 50,$ and $25$ draws. Each time
we recorded the posterior samples of $\nu$.

We found that we could consistently and precisely recover median posterior
estimates of $\nu$ near the true value of $\nu$ with large samples ($\ge 200$)
(Fig.~\ref{fig:sim-nu} upper panels). At smaller samples we could still
usually qualitatively distinguish heavy from not-heavy tails, but the model
tended to underestimate how heavy the tails were. At the same time, at smaller
sample sizes, the model tended to overestimate how large the scale parameter
was (Fig.~\ref{fig:sim-nu} lower panels).

\subsection{Heavy-tailed Gompertz model simulations}

In the second part of our simulation testing, we tested the ability of the
heavy-tailed Gompertz model to obtain unbiased parameter estimates when the
process noise was chosen so that appropriate tail events were present. To
generate these process deviations for the $\nu = 3$ and $\nu = 5$ scenarios,
we repeatedly drew proposed candidate process deviations and estimated the
central tendency, scale, and $\nu$ values each time. We recorded when
$\hat{\nu}$ (median of the posterior) was within $0.2$ CVs (coefficient of
variations) of the true $\nu$ value and used this set of random seed values in
our Gompertz simulation. The following simplified R code illustrates this
procedure (the actual code is available at
\url{https://github.com/seananderson/heavy-tails}):

\begin{footnotesize}
\begin{verbatim}
get_effective_nu_seeds <- function(nu_true = 5, cv = 0.2, N = 50, seed_N = 20) {
  # nu_true: The true nu value
  # cv:      The permitted effective nu coefficient of variation
  # N:       The length of time series
  # seed_N:  The number of seed values to generate
  seeds <- numeric(length = seed_N)
  seed_value <- 0
  for (i in seq_len(seed_N)) {
    nu_close <- FALSE
    while (!nu_close) {
      seed_value <- seed_value + 1
      set.seed(i)
      y <- rt(N, df = nu_true)
      sm <- rstan::stan(... # fit the Stan model here
      med_nu_hat <- median(rstan::extract(sm, pars = "nu")[[1]])
      if (med_nu_hat > (nu_true - cv) & med_nu_hat < (nu_true + cv)) {
        nu_close <- TRUE
        seeds[i] <- seed_value
      }
    }
  }
  seeds
}
nu_3_seeds_N50 <- get_effective_nu_seeds(nu_true = 3)
nu_5_seeds_N50 <- get_effective_nu_seeds(nu_true = 5)
\end{verbatim}
\end{footnotesize}

We then fit our Gompertz models to the simulated datasets with all parameters
(except $\nu$) set near the median values estimated in the GPDD. We repeated
this with $50$ and $100$ samples without observation error, $50$ samples with
observation error ($\sigma_\mathrm{obs} = 0.2$), and $50$ samples with the
same observation error and a Gompertz model that allowed for correctly
specified observation error magnitude. Our results indicate that the Gompertz
model can recapture the true value of $\nu$ when the process noise was chosen
so that appropriate tail events were present (Fig.~\ref{fig:sim-prob} upper
panels). The addition of observation error caused the model to tend to
underestimate the degree of heavy-tailedness. Fitting a model with correctly
specified observation error did not make substantial improvements to model
bias (Fig.~\ref{fig:sim-prob}).

%(Figs~\ref{fig:sim-gompertz} and \ref{fig:sim-gompertz-boxplots}, red and
%green symbols in the top rows). Likewise, the other Gompertz parameters were
%estimated without any systematic bias (Figs~\ref{fig:sim-gompertz} and
%\ref{fig:sim-gompertz-boxplots}, red and green symbols). , overestimate the
%magnitude of process noise, somewhat overestimate $\lambda$, and overestimate
%density dependence (blue symbols in Figs~\ref{fig:sim-gompertz} and
%\ref{fig:sim-gompertz-boxplots}). The overestimation of density dependence
%with observation error is a known phenomenon \citep{knape2012}. Fitting a
%model with correctly specified observation error made marginal improvements
%to model bias (purple symbols in Figs~\ref{fig:sim-gompertz} and
%\ref{fig:sim-gompertz-boxplots}).

When converting the posterior distributions of $\nu$ into Pr($\nu < 10$), the
models distinguished heavy and not-heavy tails reasonably well
(Fig.~\ref{fig:sim-prob} lower panels). Without observation error, and using a
probability of $0.5$ as a threshold, the model correctly classified all
simulated systems with normally distributed process noise as not heavy tailed.
The model would have miscategorized only one of $40$ simulations at $\nu = 5$
across simulated populations with $50$ or $100$ time steps
(Fig.~\ref{fig:sim-prob}, scenarios 1 and 2 in lower row, second panel from
left). The model would have correctly categorized all cases where the process
noise was not heavy tailed (Fig.~\ref{fig:sim-prob} bottom-right panel) and
all cases where $\nu = 3$ and there was not observation error. With $0.2$
standard deviations of observation error, the model still categorized
\obsErrorNuFivePerc\% of cases as heavy tailed when $\nu = 5$ and all but one
case when $\nu = 3$. Allowing for observation error made little improvement to
the detection of heavy tails. Therefore, we chose to focus on the simpler
model without observation error in the main text, particularly given that the
true magnitude of observation error was unknown in the empirical data.

\section{Modelling covariates of heavy-tailed dynamics}

We fit a multilevel beta regression model to the predicted probability of
heavy tails, Pr($\nu < 10$), to investigate potential covariates of
heavy-tailed dynamics. The beta distribution is useful when response data
range on a continuous scale between zero and one \citep{ferrari2004}. We used
a logit link function as is typically used in logistic regression. The model
was as follows:

\begin{align*}
\mathrm{Pr}(\nu_i < 10) &\sim \mathrm{Beta}(A_i, B_i)\\
\mu_i &= \mathrm{logit}^{-1}(\alpha
  + \alpha^\mathrm{class}_{j[i]}
  + \alpha^\mathrm{order}_{k[i]}
  + \alpha^\mathrm{species}_{l[i]}
  + X_i \beta),
  \: \text{for } i = 1, \dots, 617\\
A_i &= \phi_\mathrm{disp} \mu_i\\
B_i &= \phi_\mathrm{disp} (1 - \mu_i)\\
\alpha^\mathrm{class}_j &\sim
  \mathrm{Normal}(0, \sigma^2_{\alpha \; \mathrm{class}}),
  \: \text{for } j = 1, \dots, 6\\
\alpha^\mathrm{order}_k &\sim
  \mathrm{Normal}(0, \sigma^2_{\alpha \; \mathrm{order}}),
  \: \text{for } k = 1, \dots, 38\\
\alpha^\mathrm{species}_l &\sim
  \mathrm{Normal}(0, \sigma^2_{\alpha \; \mathrm{species}}),
  \: \text{for } l = 1, \dots, 301,
\end{align*}
where $A$ and $B$ represent the beta distribution shape parameters; $\mu_i$
represents the predicted value for population $i$, class $j$, order $k$, and
species $l$; $\phi_\mathrm{disp}$ represents the dispersion parameter; and
$X_i$ represents a vector of predictors (such as lifespan) for population $i$
with associated $\beta$ coefficients. The intercepts are allowed to vary from
the overall intercept $\alpha$ by taxonomic class ($\alpha^\mathrm{class}_j$),
taxonomic order ($\alpha^\mathrm{order}_k$), and species
($\alpha^\mathrm{species}_l$) with standard deviations $\sigma_{\alpha \;
  \mathrm{class}}$, $\sigma_{\alpha \; \mathrm{order}}$, and $\sigma_{\alpha
  \; \mathrm{species}}$. Where possible, we also allowed for error
distributions around the predictors by incorporating the standard deviation of
the posterior samples for the Gompertz parameters $\lambda$, $b$, and $\log
\sigma$ around the mean point value as normal distributions (not shown in the
above equation).

We log transformed $\sigma$, time-series length, and lifespan to match the way
they are visually represented in Fig.~\ref{fig:correlates} and to make the
relationship approximately linear on the logit-transformed response scale. All
input variables were standardized by subtracting their mean and dividing by
two standard deviations to make their coefficients comparable in magnitude
\citep{gelman2008c}. We excluded body length as a covariate because it was
highly correlated with lifespan, and lifespan exhibited more overlap across
taxonomy than body length. Lifespan is also more directly related to time and
potential mechanisms driving black-swan dynamics.

We incorporated weakly informative priors into our model: $\mathrm{Cauchy}(0,
10)$ on the global intercept $\alpha$, $\mathrm{Half\mhyphen Cauchy}(0, 2.5)$
on all standard deviation parameters, $\mathrm{Half\mhyphen Cauchy}(0, 10)$ on
the dispersion parameter $\phi_\mathrm{disp}$, and $\mathrm{Cauchy}(0, 2.5)$
on all other parameters \citep{gelman2006c, gelman2008d}. Compared to normal
priors, the Cauchy priors concentrate more probability density around expected
parameter values while allowing for a higher probability density far into the
tails, thereby allowing the data to dominate the posterior more strongly if it
disagrees with the prior. Our conclusions were not qualitatively changed by
using uniform priors. We fit our models with 5000 total iterations per chain,
2500 warm-up iterations, four chains, and discarding every second sample to
save memory. We checked for chain convergence visually and with the same
criteria as before ($\widehat{R} < 1.05$ and $n_\mathrm{eff} >200$ for all
parameters).

To derive taxonomic-order-level estimates of the probability of heavy tails
accounting for time-series length (Fig.~\ref{fig:posteriors}a), we fit a
separate multilevel model with the same structure but with only $\log$
time-series length as a predictor. (In this case, we did not want to control
for intrinsic population characteristics such as density dependence.) Since
our predictors were centered by subtracting their mean value, we obtained
order-level estimates of the probability of heavy tails at mean log
time-series length by adding the posteriors for $\alpha$,
$\alpha^\mathrm{class}_j$, and $\alpha^\mathrm{order}_k$.

\section{Additional acknowledgements}

Many of the silhouette images used in Figs~\ref{fig:nu-coefs},
\ref{fig:correlates} and \ref{fig:posteriors} were obtained from
\texttt{phylopic.org} under Creative Commons licenses. We vectorized the
salmon in Fig.~\ref{fig:nu-coefs} and Fig.~\ref{fig:correlates} ourselves. The
bird in these figures was obtained from \texttt{phylopic.org} under a Creative
Commons Attribution 3.0 Unported license with credit to Jean-Raphaël
Guillaumin {[}photography{]} and T. Michael Keesey {[}vectorization{]}). The
silhouettes in Fig.~\ref{fig:posteriors} were obtained from the following
sources (metadata obtained with the help of the rphylopic R package,
\url{https://github.com/sckott/rphylopic}):

\LTcapwidth=\textwidth
%% \singlespacing
\begin{footnotesize}
\begin{longtable}{>{\RaggedRight}m{3.2cm}>{\RaggedRight}p{6.5cm}>{\RaggedRight}p{5.0cm}}
%\caption{Phylopic credits}\\
\toprule
% latex table generated in R 3.1.2 by xtable 1.7-4 package
Taxonomic order & Credit & License URL \\ 
  \midrule
Salmoniformes & Servien (vectorized by T. Michael Keesey) & \url{http://creativecommons.org/licenses/by-sa/3.0/} \\ 
  Gadiformes &  & \url{http://creativecommons.org/publicdomain/mark/1.0/} \\ 
  Perciformes & Ellen Edmonson and Hugh Chrisp (vectorized by T. Michael Keesey) & \url{http://creativecommons.org/publicdomain/mark/1.0/} \\ 
  Pleuronectiformes & Tony Ayling (vectorized by T. Michael Keesey) & \url{http://creativecommons.org/licenses/by-sa/3.0/} \\ 
  Lepidoptera & Curtis (modified by T. Michael Keesey) & \url{http://creativecommons.org/publicdomain/mark/1.0/} \\ 
  Rodentia & Mattia Menchetti & \url{http://creativecommons.org/publicdomain/zero/1.0/} \\ 
  Carnivora & Brian Gratwicke (photo) and T. Michael Keesey (vectorization) & \url{http://creativecommons.org/licenses/by/3.0/} \\ 
  Lagomorpha & Sarah Werning & \url{http://creativecommons.org/licenses/by/3.0/} \\ 
  Coleoptera & Crystal Maier & \url{http://creativecommons.org/licenses/by/3.0/} \\ 
  Odonata & Gareth Monger & \url{http://creativecommons.org/licenses/by/3.0/} \\ 
  Passeriformes & Michael Scroggie & \url{http://creativecommons.org/publicdomain/zero/1.0/} \\ 
  Anseriformes & Sharon Wegner-Larsen & \url{http://creativecommons.org/publicdomain/zero/1.0/} \\ 
  Artiodactyla & Jan A. Venter, Herbert H. T. Prins, David A. Balfour and Rob Slotow (vectorized by T. Michael Keesey) & \url{http://creativecommons.org/licenses/by/3.0/} \\ 
  Diptera & Gareth Monger & \url{http://creativecommons.org/licenses/by/3.0/} \\ 
  Charadriiformes & JJ Harrison (vectorized by T. Michael Keesey) & \url{http://creativecommons.org/licenses/by-sa/3.0/} \\ 
  Hemiptera & T. Michael Keesey & \url{http://creativecommons.org/publicdomain/zero/1.0/} \\ 
  Falconiformes & Liftarn & \url{http://creativecommons.org/licenses/by-sa/3.0/} \\ 
  Galliformes & Steven Traver & \url{http://creativecommons.org/publicdomain/zero/1.0/} \\ 
   \bottomrule

\label{tab:phylopic}
\end{longtable}
\end{footnotesize}
%% \onehalfspacing


%\baselinestretch}{\tighttextstretch}
\normalsize
\bibliographystyle{apalike}
\bibliography{/Users/seananderson/Dropbox/tex/jshort,/Users/seananderson/Dropbox/tex/ref3}
%\baselinestretch}{\textstretch}
\normalsize

% ------------------------------
% Supplemental Tables
% ------------------------------

\clearpage
%\thetable}{S\arabic{table}}
%table}{0}

\begin{table}
\begin{footnotesize}

\caption[Summary statistics for the filtered Global Population Dynamics
  Database time series arranged by taxonomic class.]{Summary statistics for the filtered Global Population Dynamics
  Database time series arranged by taxonomic class. Columns are: number of
  populations, number of taxonomic orders, numbers of species, median time
  series length, total number of interpolated time steps, total number of
  substituted zeros, and total number of time steps.}

\smallskip
\begin{tabular}{lrrrrrrrr}
\toprule
% latex table generated in R 3.1.2 by xtable 1.7-4 package
% Sat Nov  8 11:58:08 2014
Taxonomic class & Populations & Orders & Species & Median length & Interpolated pts & Zeros pts & Total pts \\ 
  \midrule
Aves & 191 &  15 & 112 &  27 &  68 &  32 & 6160 \\ 
  Insecta & 182 &   7 &  91 &  25 &  26 &  55 & 4812 \\ 
  Mammalia & 125 &   8 &  51 &  28 &  18 &  21 & 4027 \\ 
  Osteichthyes & 108 &   6 &  35 &  26 &  13 &   3 & 3310 \\ 
  Chondrichthyes &   1 &   1 &   1 &  20 &   1 &   0 &  20 \\ 
  Crustacea &   1 &   1 &   1 &  33 &   0 &   0 &  33 \\ 
  Gastropoda &   1 &   1 &   1 &  21 &   0 &   0 &  21 \\ 
   \bottomrule

\label{tab:stats}
\end{tabular}
\end{footnotesize}
\end{table}

\clearpage

\LTcapwidth=\textwidth
\bibpunct{}{}{;}{a}{}{;}

%% \singlespacing
\begin{footnotesize}
\begin{longtable}{>{\RaggedRight}m{1.5cm}>{\RaggedRight}p{4.3cm}>{\RaggedRight}p{0.8cm}>{\RaggedRight}p{1.7cm}>{\RaggedRight}p{1.0cm}>{\RaggedRight}p{3.0cm}>{\RaggedRight}p{1.7cm}>{\RaggedRight}p{1.3cm}}

\caption[All populations with Pr$(\nu < 10) > 0.5$ in the base heavy-tailed
  Gompertz population dynamics model.]{All populations with Pr$(\nu < 10) > 0.5$ in the base heavy-tailed
  Gompertz population dynamics model. Shown are the log abundance time series,
  population descriptions, Global Population Dynamics Database Main IDs,
  citation for the data source or separate verification literature, a
  description of the cause of the black swan events (if known), the
  probability of heavy tails as calculated by our model, and median estimate
  of $\nu$ from our model with 90\% quantile credible intervals indicated in
  parentheses. Red dots on the time series indicate downward black-swan events
  and blue values indicate upward black-swan events that have a $10 \cdot
  10^{-4}$ probability or less of occurring if the population dynamics were
  explained by a Gompertz model with normally distributed process noise with a
  standard deviation equal to the scale parameter in the fitted t
  distribution.}\\

\toprule
% latex table generated in R 3.1.2 by xtable 1.7-4 package
Time series & Population & ID & Citation & Description & Pr($\nu < 10$) & $\widehat{\nu}$ \\ 
  \midrule
\includegraphics[width=1.7cm]{blackswans/analysis/sparks/6528.pdf} & Shag, \textit{Phalacrocorax aristotelis}, Farne Islands, Northumberland & 6528 & \citep{potts1980} & Red tide event combined with low productivity due to overcrowding & 1.00 & 2 (2--4) \\ 
  \includegraphics[width=1.7cm]{blackswans/analysis/sparks/7115.pdf} & South African fur seal, \textit{Arctocephalus pusillus}, South Africa & 7115 & \citep{shaughnessy1982} & Harvesting and predation changes & 1.00 & 2 (2--4) \\ 
  \includegraphics[width=1.7cm]{blackswans/analysis/sparks/10128.pdf} & Red grouse, \textit{Lagopus lagopus scoticus}, Scotland - un-named area & 10128 & \citep{potts1984} & Environment- and parisite-caused cycles & 1.00 & 3 (2--4) \\ 
  \includegraphics[width=1.7cm]{blackswans/analysis/sparks/9382.pdf} & Pine looper or Bordered white, \textit{Bupalus piniaria}, Kessock & 9382 & \citep{broekhuizen1993} & Unknown, but sampling intensity was decreasing & 1.00 & 3 (2--5) \\ 
  \includegraphics[width=1.7cm]{blackswans/analysis/sparks/10127.pdf} & Red grouse, \textit{Lagopus lagopus scoticus}, Scotland - un-named area & 10127 & \citep{potts1984} & Environment- and parisite-caused cycles & 0.99 & 3 (2--5) \\ 
  \includegraphics[width=1.7cm]{blackswans/analysis/sparks/10007.pdf} & Water vole, \textit{Arvicola terrestris}, Le Pont & 10007 & \citep{saucy1994} & Predator-environment cycle interactions & 1.00 & 3 (2--5) \\ 
  \includegraphics[width=1.7cm]{blackswans/analysis/sparks/20579.pdf} & Grey heron, \textit{Ardea cinerea}, Southern Britain & 20579 & \citep{stafford1971} & Severe winter & 0.98 & 3 (2--7) \\ 
  \includegraphics[width=1.7cm]{blackswans/analysis/sparks/9655.pdf} & Flea beetle, \textit{Chaetocnoma concinna}, Finland & 9655 & \citep{markkula1965} & Cannot locate original source & 0.99 & 3 (2--6) \\ 
  \includegraphics[width=1.7cm]{blackswans/analysis/sparks/1235.pdf} & Wren, \textit{Troglodytes troglodytes}, Eastern Wood, Bookham Common & 1235 & \citep{newton1998} & Severe winter & 0.98 & 3 (2--8) \\ 
  \includegraphics[width=1.7cm]{blackswans/analysis/sparks/10113.pdf} & Willow grouse, \textit{Lagopus lagopus}, Northern England & 10113 & \citep{dobson1995} & Parasites and predators & 0.99 & 3 (2--6) \\ 
  \includegraphics[width=1.7cm]{blackswans/analysis/sparks/9667.pdf} & Gooseberry sawfly, \textit{Nemastus ribesii}, Finland & 9667 & \citep{markkula1965} & Cannot locate original source & 0.97 & 3 (2--8) \\ 
  \includegraphics[width=1.7cm]{blackswans/analysis/sparks/9679.pdf} & Unknown, \textit{Trioza apicalis}, Finland & 9679 & \citep{markkula1965} & Cannot locate original source & 0.83 & 3 (2--88) \\ 
  \includegraphics[width=1.7cm]{blackswans/analysis/sparks/20527.pdf} & Wandering albatross, \textit{Diomedea exulans}, Taiaroa & 20527 & \citep{robertson1998} & Unknown & 0.99 & 4 (2--7) \\ 
  \includegraphics[width=1.7cm]{blackswans/analysis/sparks/10039.pdf} & Red grouse, \textit{Lagopus lagopus scoticus}, Northern Scotland & 10039 & \citep{dobson1995} & Parasites and predators & 0.92 & 4 (2--12) \\ 
  \includegraphics[width=1.7cm]{blackswans/analysis/sparks/10162.pdf} & Red grouse, \textit{Lagopus lagopus scoticus}, Atholl Estate & 10162 & \citet{mackenzie1952} & Bad environmental conditions and overcrowding combined to create crashes & 0.91 & 5 (2--12) \\ 
  \includegraphics[width=1.7cm]{blackswans/analysis/sparks/9503.pdf} & Fisher or  Pekan, \textit{Martes pennanti}, Manitoba & 9503 & \citep{keith1963} & Unknown & 0.76 & 5 (2--87) \\ 
  \includegraphics[width=1.7cm]{blackswans/analysis/sparks/7099.pdf} & European rabbit, \textit{Oryctolagus cuniculus}, Estate 2, East Anglia & 7099 & \citep{barnes1986} & Disease outbreak followed by years of good weather & 0.78 & 5 (2--67) \\ 
  \includegraphics[width=1.7cm]{blackswans/analysis/sparks/2778.pdf} & Wheatear, \textit{Oenanthe oenanthe}, Skokholm Island & 2778 & \citep{lack1969} & Unknown, but decline noted specifically, cold winters caused some crashes & 0.67 & 5 (2--141) \\ 
  \includegraphics[width=1.7cm]{blackswans/analysis/sparks/9659.pdf} & Cabbage root fly or maggot, \textit{Delia radicum}, Finland & 9659 & \citep{markkula1965} & Cannot locate original source & 0.59 & 7 (2--161) \\ 
  \includegraphics[width=1.7cm]{blackswans/analysis/sparks/1195.pdf} & Blue jay, \textit{Cyanocitta cristata}, Robert Allerton Park & 1195 & \citep{kendeigh1982} & Unknown & 0.60 & 7 (2--149) \\ 
  \includegraphics[width=1.7cm]{blackswans/analysis/sparks/5019.pdf} & Barbary macaque, \textit{Macaca sylvanus}, Queens Gate & 5019 & \citep{fa1984} & Cannot locate original source & 0.61 & 7 (2--132) \\ 
  \includegraphics[width=1.7cm]{blackswans/analysis/sparks/9953.pdf} & Rock ptarmigan, \textit{Lagopus mutus}, Iceland & 9953 & \citep{clarke1885,williams1954} & Severe winters & 0.58 & 7 (2--172) \\ 
  \includegraphics[width=1.7cm]{blackswans/analysis/sparks/6548.pdf} & Lesser-spotted dogfish, \textit{Scyliorhinus caniculus}, North Sea & 6548 & \citep{heessen1996} & Unknown, not specifically mentioned & 0.53 & 8 (2--190) \\ 
  \includegraphics[width=1.7cm]{blackswans/analysis/sparks/20546.pdf} & American red fox, \textit{Vulpes fulva}, Labrador & 20546 & \citep{dancona1954,lindstrom1994} & Predator-prey cycles & 0.57 & 9 (3--65) \\ 
  \includegraphics[width=1.7cm]{blackswans/analysis/sparks/9470.pdf} & Ruffed grouse, \textit{Bonasa umbellus}, Connecticut & 9470 & \citep{keith1963} & Unknown & 0.53 & 9 (3--158) \\ 
   \bottomrule

\label{tab:causes-supp}
\end{longtable}
\end{footnotesize}
%% \onehalfspacing

% ------------------------------
% Supplemental Figures
% ------------------------------

%\thefigure}{S\arabic{figure}}
%figure}{0}

\begin{centering}
\clearpage
\includegraphics[width=\textwidth]{blackswans/analysis/all-clean-ts-mammals.pdf}\\
Figure~\ref{fig:all-ts} (mammals) continued on next page \ldots

\clearpage
\includegraphics[width=\textwidth]{blackswans/analysis/all-clean-ts-birds.pdf}\\
Figure~\ref{fig:all-ts} (birds) continued on next page \ldots

\clearpage
\includegraphics[width=\textwidth]{blackswans/analysis/all-clean-ts-insects.pdf}\\
Figure~\ref{fig:all-ts} (insects) continued on next page \ldots

\end{centering}

\begin{figure}[htbp]
\begin{center}
\includegraphics[width=\textwidth]{blackswans/analysis/all-clean-ts-fishes-others.pdf}

\caption[All filtered time series used in our analysis.]{(fishes, crustaceans,
  gastropods, sharks). All filtered time series used in our analysis. The
  abundances are shown on a log10 vertical axis. Throughout this figure, red
  dots indicate values that were interpolated and blue dots indicate values
  that were recorded as zero but were set to the next lowest observed
  abundance. Numbers before each species name are the GPDD Main ID numbers.}

\label{fig:all-ts}
\end{center}
\end{figure}

\clearpage

\begin{figure}[htbp]
\begin{center}
\includegraphics[width=0.8\textwidth]{blackswans/analysis/priors-gomp-base.pdf}

\caption[Probability density of the Bayesian priors for the Gompertz models.]{
  Probability density of the Bayesian priors for the Gompertz models. From
  left to right and then top to bottom: (1) per capita growth rate at
  $\log$(abundance) = $0$: $\lambda \sim \mathrm{Normal}(0, 10^2)$; (2) scale
  parameter of t-distribution process noise: $\sigma \sim \mathrm{Half\mhyphen
    Cauchy} (0, 2.5)$; (3) t-distribution degrees of freedom parameter: $\nu
  \sim \mathrm{Truncated\mhyphen Exponential}(0.01, \mathrm{min.} = 2)$; (4)
  AR1 correlation coefficient of residuals: $\phi \sim \mathrm{Truncated
    \mhyphen Normal}(0, 1, \mathrm{min.} = -1, \mathrm{max.} = 1)$. Not shown
  is $b$, the density dependence parameter: $b \sim \mathrm{Uniform}(-1, 2)$.
  The $\nu$ panel also shows two alternative priors: a weaker prior $\nu \sim
  \mathrm{Truncated\mhyphen Exponential}(0.005, \mathrm{min.} = 2)$, and a
  stronger prior $\nu \sim \mathrm{Truncated\mhyphen Exponential}(0.02,
  \mathrm{min.} = 2)$. The inset panel shows the same data but with a
  log-transformed x axis. Note that the base and weaker priors are relatively
  flat within the region of $\nu < 20$ that we are concerned with. }

\label{fig:priors}
\end{center}
\end{figure}

\clearpage

\begin{figure}[htbp]
\begin{center}
\includegraphics[width=\textwidth]{blackswans/analysis/gomp-comparison.pdf}

\caption[Estimates of $\nu$ from alternative models plotted against the base
Gompertz model estimates of $\nu$.]{Estimates of $\nu$ from alternative models
  plotted against the base Gompertz model estimates of $\nu$. Shown are
  medians of the posterior (dots) and 50\% credible intervals (segments). The
  diagonal line indicates a one-to-one relationship. Different colours
  indicate various taxonomic classes. The grey-shaded regions indicate regions
  of disagreement if $\nu = 10$ is taken as a threshold of heavy-tailed
  dynamics. The Gompertz observation error model assumes a fixed standard
  deviation of observation error of $0.2$ on a log scale.}

\label{fig:alt}
\end{center}
\end{figure}

\clearpage

\begin{figure}[htbp]
\begin{center}
\includegraphics[width=\textwidth]{blackswans/analysis/gomp-prior-comparison.pdf}

\caption[Estimates of $\nu$ from Gompertz models with alternative priors on
$\nu$.]{Estimates of $\nu$ from Gompertz models with alternative priors on
  $\nu$. Shown are medians of the posterior (dots) and 50\% credible intervals
  (segments). The diagonal line indicates a one-to-one relationship. Different
  colours indicate various taxonomic classes. The grey-shaded regions indicate
  regions of disagreement if $\nu = 10$ is taken as a threshold of
  heavy-tailed dynamics. The base, weaker, and stronger priors on $\nu$ are
  illustrated in Fig.~\ref{fig:priors}. In general, the estimates are nearly
  identical in cases where the data are informative about low values of $\nu$.
  When the data are less informative about low values of $\nu$, the prior can
  slightly pull the estimates of $\nu$ towards higher or lower values.}

\label{fig:alt-priors}
\end{center}
\end{figure}

\clearpage

\begin{figure}[htbp]
\begin{center}
\includegraphics[width=0.8\textwidth]{blackswans/analysis/t-dist-sampling-sim-prior-exp0point01.pdf}
\includegraphics[width=0.8\textwidth]{blackswans/analysis/t-dist-sampling-sim-sigma-prior-exp0point01.pdf}

\caption[Testing the ability to estimate $\nu$  and the scale parameter of the
process deviations for a given number of samples drawn from a distribution
with a given true $\nu$ value.]{Testing the ability to estimate $\nu$ (top
  panels) and the scale parameter of the process deviations (bottom panels)
  for a given number of samples (columns) drawn from a distribution with a
  given true $\nu$ value (rows). The red lines indicate the true population
  value. When a small number of samples are drawn there may not be samples
  sufficiently far into the tails to recapture the true $\nu$ value; however,
  heavy tails are still distinguished from normal tails in most cases, even
  with only 25 or 50 samples.}

\label{fig:sim-nu}
\end{center}
\end{figure}

\clearpage

\begin{figure}[htbp]
\begin{center}
\includegraphics[width=\textwidth]{blackswans/analysis/sim-gompertz-median-dist.pdf}
\includegraphics[width=\textwidth]{blackswans/analysis/sim-gompertz-p10.pdf}

\caption[Simulation testing the Gompertz estimation model when the process
deviation draws were chosen so that $\nu$ could be estimated close to the true
value outside the full population model (``effective $\nu$'' within a CV of
0.2 of specified $\nu$).]{Simulation testing the Gompertz estimation model
  when the process deviation draws were chosen so that $\nu$ could be
  estimated close to the true value outside the full population model
  (``effective $\nu$'' within a CV of 0.2 of specified $\nu$). Upper panels
  show the distribution of median $\widehat{\nu}$ across 20 simulation runs.
  Lower panels show the distribution of Pr($\nu < 10$) across 20 simulation
  runs. We ran the simulations across three population (``true'') $\nu$ values
  (3, 5, and $1\cdot 10^9$, i.e.\ approximately normal) and four scenarios:
  (1) 100 time steps and no observation error, (2) 50 time steps and no
  observation error, (3) 50 time steps and observation error drawn from
  $\mathrm{Normal} (0,
  0.2^2)$ but ignored, and (4) 50 time steps with observation error in which
    the quantity of observation error was assumed known. Within each scenario
    the dots represent stochastic draws from the true population distributions
    combined with model fits. Underlayed boxplots show the median,
    interquartile range, and $1.5$ times the interquartile range. }

\label{fig:sim-prob}
\end{center}
\end{figure}

\clearpage

\noindent
Example Stan code for a heavy-tailed Gompertz model with AR1 correlated
residuals and a specified level of observation error. The specific code for
used for the various models in our analysis is available at
\url{https://github.com/seananderson/heavy-tails}.

%% \begin{spacing}{1.15}
\begin{footnotesize}
\begin{verbatim}
data {
  int<lower=3> N;              // number of observations
  vector[N] y;                 // vector to hold ln abundance observations
  real<lower=0> nu_rate;       // rate parameter for nu exponential prior
}
parameters {
  real lambda;                 // Gompertz growth rate parameter
  real<lower=-1, upper=2> b;   // Gompertz density dependence parameter
  real<lower=0> sigma_proc;    // process noise scale parameter
  real<lower=2> nu;            // t-distribution degrees of freedom
  real<lower=-1, upper=1> phi; // AR1 parameter
  vector[N] U;                 // unobserved states
  real<lower=0> sigma_obs;     // specified observation error SD
}
transformed parameters {
  vector[N] epsilon;           // error terms
  epsilon[1] <- 0;
  for (i in 2:N) {
    epsilon[i] <- U[i] - (lambda + b * U[i - 1])
                       - (phi * epsilon[i - 1]);
  }
}
model {
  // priors:
  nu ~ exponential(nu_rate);
  lambda ~ normal(0, 10);
  sigma_proc ~ cauchy(0, 2.5);
  phi ~ normal(0, 1);
  // data model:
  for (i in 2:N) {
    U[i] ~ student_t(nu,
                     lambda + b * U[i - 1]
                     + phi * epsilon[i - 1],
                     sigma_proc);
  }
  y ~ normal(U, sigma_obs);
}
\end{verbatim}
\end{footnotesize}

\clearpage
\noindent
Stan code for the multilevel beta regression:
\begin{footnotesize}
\verbatiminput{blackswans/analysis/betareg4.stan}
\end{footnotesize}

\clearpage

\noindent
The GPDD IDs used in our analysis.

%\baselinestretch}{\tighttextstretch}
\normalsize
\begin{footnotesize}
\noindent
{\tt
1 3 4 5 6 7 8 9 10 11 12 13 14 15 16 17 18 44 45 46 47 58 61 64 1149 1150 1153
1157 1159 1160 1162 1163 1165 1166 1168 1169 1170 1173 1174 1177 1179 1184 1185
1188 1189 1190 1195 1196 1197 1199 1200 1201 1202 1203 1204 1205 1206 1217 1227
1228 1229 1233 1234 1235 1237 1238 1239 1240 1243 1244 1247 1342 1377 1522 1523
1524 1525 1534 1602 1613 1618 1633 1660 1663 1664 1667 1669 1670 1671 1674 1682
1683 1792 1826 1829 1830 1831 1865 1866 1868 1869 1870 1875 1876 1880 1881 1883
1885 1886 1887 1888 1893 1894 1927 1964 1965 1966 1968 1970 1971 1973 1974 1976
1981 1982 1983 1986 1987 1991 1992 1993 1994 1998 1999 2003 2004 2005 2006 2007
2012 2013 2015 2016 2017 2018 2019 2020 2024 2025 2026 2027 2028 2031 2032 2033
2034 2066 2721 2722 2726 2732 2735 2736 2757 2758 2759 2770 2771 2772 2774 2775
2777 2778 2781 2829 2844 2857 2867 2869 2887 2903 2915 2974 2976 2991 3001 3003
3017 3051 3056 3059 3068 3214 3216 3218 3233 3249 3251 3253 3260 3265 3283 3356
3358 3360 3378 3442 3466 3468 3470 3477 3482 3508 3521 3625 3627 3639 3664 3673
3676 3678 3680 3706 3708 3716 3774 3776 3784 3795 3799 3811 3827 3829 3838 3840
3853 3866 3882 5019 5020 5032 5034 5035 5039 6057 6144 6527 6528 6529 6530 6532
6533 6534 6535 6536 6537 6539 6541 6542 6547 6548 6549 6550 6553 6554 6555 6556
6558 6560 6561 6562 6564 6565 6567 6568 6569 6570 6581 6582 6583 6633 6673 6674
6675 6676 6677 6678 6681 6683 6684 6685 6686 6687 6688 6770 6865 6867 6868 6869
6870 6876 6882 6885 6889 6890 6902 6904 6917 6920 6921 6922 6939 6940 6973 7048
7052 7053 7054 7060 7061 7067 7088 7089 7091 7092 7093 7094 7098 7099 7101 7102
7115 7116 9191 9192 9194 9195 9196 9200 9211 9215 9216 9217 9218 9219 9220 9221
9222 9223 9224 9225 9232 9308 9309 9330 9331 9381 9382 9393 9436 9437 9438 9439
9440 9441 9442 9443 9444 9445 9446 9468 9469 9470 9472 9477 9486 9488 9489 9490
9491 9492 9500 9501 9502 9503 9506 9515 9517 9518 9519 9586 9587 9606 9611 9612
9639 9641 9642 9644 9646 9647 9648 9650 9652 9654 9655 9656 9657 9658 9659 9661
9662 9663 9665 9667 9668 9669 9672 9673 9674 9675 9676 9677 9678 9679 9680 9681
9682 9688 9689 9690 9691 9793 9794 9795 9796 9797 9835 9836 9893 9894 9895 9896
9897 9898 9899 9900 9901 9902 9903 9904 9905 9907 9919 9921 9932 9933 9934 9936
9938 9948 9949 9950 9951 9953 9990 9991 9993 9994 9995 9997 9998 9999 10000
10001 10002 10005 10006 10007 10008 10009 10010 10011 10012 10013 10029 10030
10031 10036 10039 10040 10041 10042 10044 10045 10046 10047 10048 10049 10050
10051 10053 10054 10055 10060 10061 10063 10065 10070 10071 10085 10088 10089
10090 10092 10093 10094 10096 10097 10098 10099 10100 10101 10110 10111 10112
10113 10114 10117 10118 10120 10121 10122 10123 10124 10125 10127 10128 10131
10134 10136 10137 10140 10141 10142 10143 10144 10145 10149 10153 10156 10158
10159 10160 10161 10162 10163 10164 10165 20527 20530 20532 20534 20535 20536
20537 20539 20540 20541 20542 20543 20544 20546 20547 20548 20549 20550 20551
20552 20553 20555 20577 20578 20579 20580 20581 20582 20583 20587 20626 20628
20634 20635 20636 20639 20649 20650 20651 20652 20653 20654 20655 20656 20657
20658 20659 20660 20662 20663

}
\end{footnotesize}
%\baselinestretch}{\textstretch}
\normalsize
%% \end{spacing}

