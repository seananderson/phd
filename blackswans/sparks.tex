\LTcapwidth=\textwidth
\bibpunct{}{}{;}{a}{}{;}

\begin{small}
\begin{longtable}{>{\RaggedRight}m{2.0cm}>{\RaggedRight}p{3.0cm}>{\RaggedRight}p{7.0cm}>{\RaggedRight}p{2.0cm}}

\caption[Example population dynamic black swans from the Global Population
  Dynamics Database and a description of their causes.]{Example population dynamic black swans from the Global Population
  Dynamics Database and a description of their causes. Red and blue dots
  indicate downward and upward events that have a $10 \cdot 10^{-4}$ probability or less of
  occurring if the population dynamics were explained by a Gompertz model
  with normally distributed process noise. These populations are
  a sample from the heavy-tailed populations we could verify
  (Table~\ref{tab:causes-supp}).}\\

\toprule
Time series (log scale) & Population & Black swan description & Reference \\
\midrule

\includegraphics[width=2cm]{blackswans/analysis/sparks/6528} &
Shag,
\textit{Phalacrocorax aristotelis},
UK &
Shortage of nest sites reduced productivity; red-tide event in 1968 caused
extreme mortality; no longer a nest shortage; population rapidly increased &
\citep{potts1980}\\

\includegraphics[width=2cm]{blackswans/analysis/sparks/10007} &
Water vole,
\textit{Arvicola terrestris},
UK &
Short-term population cycles from predator interactions combined with long-term
environmental cycle caused sharp downswing  &
\citep{saucy1994}\\

\includegraphics[width=2cm]{blackswans/analysis/sparks/7115} &
Fur seal,
\textit{Arctocephalus pusillus},
South Africa &
Strong decreases in harvesting, loss of predators, and diamond mining
regulations reducing human traffic caused sharp upswings  &
\citep{shaughnessy1982}\\

\includegraphics[width=2cm]{blackswans/analysis/sparks/10113} &
Willow grouse,
\textit{Lagopus lagopus},
UK &
Parasite and predation effects interacted to cause low years  &
\citep{dobson1995}\\

\includegraphics[width=2cm]{blackswans/analysis/sparks/10162} &
Red grouse,
\textit{Lagopus lagopus scoticus},
UK &
Good environmental conditions produced high numbers and vulnerable populations;
bad conditions and overcrowding combined to create crashes  &
\citep{mackenzie1952}\\

\includegraphics[width=2cm]{blackswans/analysis/sparks/1235} &
Wren,
\textit{Troglodytes troglodytes},
UK &
Severe winters where food was buried under snow caused population crash &
\citep{newton1998} \\

\includegraphics[width=2cm]{blackswans/analysis/sparks/20579} &
Grey heron,
\textit{Ardea cinerea},
UK &
Severe winters in 1929, 1940--1942, and 1962--1963; 1963 event so severe that
recovery took three times longer than expected &
\citep{stafford1971} \\

%\includegraphics[width=2cm]{blackswans/analysis/sparks/20580} &
%Chamois, \textit{Rupicapra rupicapra}, Switzerland &
% &
%\citep{brook2006a}\\

%\includegraphics[width=2cm]{blackswans/analysis/sparks/5019} &
%Barbary macaque,  &
% &
%REF\\

%\includegraphics[width=2cm]{blackswans/analysis/sparks/9675} &
%Carrot fly (\textit{Psila rosae}, Finland &
% &
%\citep{markkula1965}\\

%\includegraphics[width=2cm]{blackswans/analysis/sparks/10139} &
%Grey heron (\textit{Ardea cinerea}), UK &
% &
%\citep{stafford1971}\\

\bottomrule
\label{tab:sparks}
\end{longtable}
\end{small}

% reset citation style:
\bibpunct{(}{)}{;}{a}{}{;}
