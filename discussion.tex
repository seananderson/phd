\chapter{General discussion}

token citation placeholder: \citep{anderson2014}

Strengths of portfolio perspective: units theories emphasizes variance conveys diversification and risk

What I showed: mean-variance problem with portfolio effects and a new way of measuring stability in metapopulations

ability of response diversity to create stable and productive metapopulation portfolios\ldots{} how efficient frontiers can illustrate inherent tradeoffs in management or conservation decisions (decreasing habitat scenario of metafolio)

Challenges and opportunities of applying portfolio theory to population dynamics\ldots{} summarize these

How we can measure synchrony in a Bayesian context\ldots{} powerful tool to separating synchrony and variability components in communities and metapopulations

Ways forward:

Black swans

synchrony modelling

applying population dynamics portfolio optimization

experimental work exploring the portfolio effect and portfolio management

role of response diversity in stabilizing populations, metapopulations, and communities

\section{Challenges}\label{challenges}

\subsection{Multidimensional ecological attributes}\label{multidimensional-ecological-attributes}

Ecological decision making is often a multidimensional problem. For example, a manager needs to balance resources available for hunting or fishing with leaving sufficient resources for ecosystem stability and function. But, financial portfolio optimization typically deals with one dimension---financial returns. Therefore, at first glance it might appear that portfolio optimization is only applicable to a narrow range of ecological decision making. On the contrary, existing approaches developed for other decision making tools can allow portfolio theory to be applied to multidimensional objectives. For instance, the roots of decision analysis (REF) \citep{keeney1976, keeney1982}, a formal method for evaluating complex decision problems, deals with decision making for one-dimensional objectives. But, decision analysis is commonly extended to multiple objectives by condensing objectives into a single dimension through multiattribute utility theory \citep{keeney1976} or presenting the results of multiple decision trees and allowing decision makers to explicitly see the trade off between multiple objectives (REF). A similar approach could be applied to ecological portfolio optimization. Furthermore, there is no reason why portfolio optimization cannot be combined with other decision making approaches (REF).

\subsection{Attributes of financial and ecological data}\label{attributes-of-financial-and-ecological-data}

Ecological and financial data differ in many fundamental ways that will affect how financial portfolio theory can be applied to ecological systems (Table~\ref{tab:data}). For example, ecological data are often of short duration, recorded at low frequency (e.g.~each year), and often contain missing values. Financial data, on the other hand, are often available at extremely high frequency (e.g.~by the second), are often available for decades, and rarely contain missing values. Econometric techniques built to manage high-frequency irregularly-spaced financial data \citep[e.g.][]{hautsch2012} may not apply to much of today's ecological data. However, these techniques may become increasingly useful as similar types of ecological data become more common \citep[e.g.~the Ocean Tracking Network,][]{cooke2011}. Another difference between financial and ecological data, is that ecological data often include considerable measurement error that adds uncertainty around the true value of ecological assets. Financial stock returns, however, reflect the trading value of a stock by definition. Therefore, to accurately apply financial portfolio optimization to ecological portfolios, we may need to adopt methods that can incorporate measurement error. Solutions may include Bayesian methods, Monte Carlo simulation, and state space modelling \citep{morgan1990}.

\section{Opportunities}\label{opportunities}

\subsection{Ecological portfolio optimization}\label{ecological-portfolio-optimization}

A central focus of financial portfolio theory is selecting an optimal weighting of assets to maximize return for a given level of risk or minimize risk for a given level of return \citep{markowitz1952}. In this paper, we have discussed optimizing ecological resource use to improve the income of ecological resource harvesters. However, we have mentioned only three examples of portfolio optimization that benefit a purely conservation outcome \citep{crowe2008, ando2012, anderson2014}.

Any ecological scenario that produces time series analogous to typical financial stock time series may be a candidate for ecological conservation portfolio optimization. Population growth rate within a metapopulation context is one possibility and will generally represent a stationary time series in the same way that the first difference of returns are frequently used in financial portfolio analysis \citep{anderson2014} (Fig.~\ref{fig:risk}). The central issue with applying conservation-oriented ecological portfolio optimization will be deciding what precisely constitutes ``investment'' and how that investment re-allocation will occur. For example, if abundance of fish stocks is considered investment how does one increase or decrease that investment through time? Perhaps these adjustments could be made through changes in harvesting limits, stock supplementation, or habitat restoration. Realistic reinvestment strategies may be taxa-, region-, or case-specific, and are so far mostly unexplored.

\subsection{Ecological risk metrics}\label{ecological-risk-metrics}

Risk assessment is a critical component to biological conservation and management \citep{burgman2005}. Conservation biology has typically used metrics of symmetric variability such as the standard deviation or the coefficient of variation of a time series \citep[e.g.][]{greene2010} and sometimes loosely referred to the result as risk; but, risk specifically refers to both the probability of an undesired event happening and the magnitude of loss associated with that event \citep[Fig.~\ref{fig:risk}]{morgan1990}. For example, if a precise outcome is unknown but can only end positively, it presents little risk. Therefore, ecological risk metrics should allow for an asymmetry in this loss function \citep{reckhow1994}. Furthermore, recent work in ecology has noted the frequency and influence of population dynamic catastrophes \citep{gerber2001, ward2007}, ecological surprises \citep{lindenmayer2010, doak2008}, counterintuitive responses of populations to management \citep{pine-iii2009}, and ecological black swans---statistically improbable events with large influence that nonetheless occur \citep{nunez2012}. Therefore, like financial risk metrics, we should consider heavier-tailed probability distributions than standard distributions such as the normal \citep{hummel2009}. The financial literature is rich with methods to rapidly assess the risk properties of time series---methods that are particularly useful when used as part of portfolio optimization \citep{rachev2008}.

Recent financial literature has focused on downside risk metrics \citep{ang2006}, which emphasize the probability of an undesired event; we see great opportunity for their application in conservation biology (Fig.~\ref{fig:risk}). A variety of downside risk metrics measure different properties of risk. Therefore, how do different metrics reflect the goals of different ``investors'' in ecological portfolios? Conservation organizations, for example, may be concerned with avoiding catastrophic single years that could influence future productivity or have downstream effects on predators or prey. They might use the probability of ruin, which measures the probability of an event worse than some threshold occurring \citep{vasicek1987} or the conditional value-at-risk (CVaR) to characterize the average magnitude of an extremely bad event \citep{rockafellar2002, sethi2012a, sethi2012b}. Resource users, on the other hand, might wish to minimize year-to-year fluctuations to ensure a stable income. Their interests could be reflected in the semideviation or semivariance, which characterizes the typical ``badness'' or severity of an event \citep{markowitz1959, sethi2012a, sethi2012b}. Moving forward, a fruitful area of research may be matching risk metrics to specific conservation-management goals \citep{sethi2012a}.

\section{Outlook}\label{outlook}

The application of portfolios concepts to ecological systems is still a young discipline and there exist many important future questions to address: For example:

\begin{enumerate}
\def\labelenumi{\arabic{enumi}.}
\item
  Across taxa, geography, and time, how pervasive is the stabilizing ecological portfolio effect and what factors affect its strength? Recent work suggests the effect may vary greatly across systems, but the general pervasiveness and the factors that promote it remain largely unclear \citep[e.g.][]{anderson2013, mellin2014}.
\item
  How can the portfolio effect and portfolio optimization inform management and conservation and in what other contexts can it be applied? For instance, how might portfolio optimization inform the debate about the advantages of forming single large or several small reserves (SLOSS)? As another example, what can portfolio optimization tell us about managing the recovery of impacted ecological systems?
\item
  How can we integrate established principles of conservation management with portfolio theory? Furthermore, what elements of portfolio theory can we integrate with traditional principles of ecological management?
\item
  How can we move ecological portfolios beyond an academic exercise to using their principles in applied management? Recent work has shown clear theoretical advantages to ecological conservation that considers MPT \citep{crowe2008, ando2012, anderson2014}, but, to our knowledge, MPT has yet to be integrated into real-word conservation planning.
\end{enumerate}

\section{Recommendations}\label{recommendations}

Given our review of ecological portfolios, we make the following recommendations:

\begin{enumerate}
\def\labelenumi{\arabic{enumi}.}
\item
  Consider whether conservation problems can fit into a portfolio framework. Simultaneously considering how management actions affect mean responses and variability (or risk) of responses is a powerful conservation management tool and can be integrated with decision analysis.
\item
  Portfolio optimization can be applied right now in cases where `investors', `assets', and `reinvestment' concepts are clear and where there are a limited number of objectives to optimize. For example, we can use portfolio optimization to set habitat conservation priorities for salmon populations under climate change uncertainty \citep{anderson2014}.
\item
  A rich area of research is exploring how we can apply portfolio optimization to ecological decision making for systems where data properties differ substantially from financial data. Another important area of applied research is determining how we can operationalize the outputs of ecological portfolio optimization into conservation decisions.
\item
  Even if not used as a research tool, portfolios provide a compelling metaphor to communicate the sometimes-abstract concept of biological diversity and its impact on risk, uncertainty, and variability to conservation managers and the public.
\end{enumerate}

\bibliographystyle{apalike}\bibliography{/Users/seananderson/Dropbox/tex/jshort,/Users/seananderson/Dropbox/tex/ref3}
