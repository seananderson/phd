%% Copyright 1998 Pepe Kubon
%%
%% `abstract.tex' --- abstract for thes-full.tex, thes-short-tex from
%%                    the `csthesis' bundle
%%
%% You are allowed to distribute this file together with all files
%% mentioned in READ.ME.
%%
%% You are not allowed to modify its contents.
%%

%%%%%%%%%%%%%%%%%%%%%%%%%%%%%%%%%%%%%%%%%%%%%%%%%
%
%       Abstract
%
%%%%%%%%%%%%%%%%%%%%%%%%%%%%%%%%%%%%%%%%%%%%%%%%

\prefacesection{Abstract}

TODO --- very rough --- from a proposal

% 350 words

In my first chapter, I review the application of financial portfolio theory to
ecology. I review the three contrasting applications: metaphors to communicate
the benefit of biological diversity, metrics to compare the benefits of
diversification across systems and time, and management tools to make robust
conservation decisions in the face of environmental uncertainty. I highlight
issues with applying portfolio theory to ecology (e.g. how we 'invest' in
financial and ecological portfolios can fundamentally differ and ecological
data presents new statistical issues) and outline key questions for the field
to address.

In my second chapter, I evaluate two methods for estimating the ecological
portfolio effect and apply these methods to metapopulation data from around the
world across moths, reef fishes, and salmon. I show an inherent bias to a
commonly applied method, outline recommendations for researchers when
estimating ecological portfolio effects, and show that the general tendency is
for stabilizing ecological portfolio effects.

In my third chapter, I show how a portfolio approach to managing population
diversity can inform conservation priorities for salmon populations under a
changing climate. My findings suggest the importance of conserving the
processes that promote thermal-tolerance diversity, such as genetic diversity,
habitat heterogeneity, and natural disturbance regimes, and demonstrate that
diverse natural portfolios may be critical for metapopulation conservation in
the face of increasing climate variability and change.

In my fourth chapter, I take the concept of heavy tails and black swans
(extreme and unexpected events) from the financial literature and ask what the
evidence is for these events in ecological abundance time series. I find strong
but rare evidence of ecological black swans across hundreds of populations from
around the world. When ecological black swans occur they tend to be associated
with extreme climate, parasites, predation, and interactions of these elements
with little relationship to life history.
