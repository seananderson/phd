\prefacesection{Abstract}

Assessing, managing, and communicating risk is fundamental to effective
ecological decision making. One promising approach is to borrow concepts
from financial portfolio management. Ecological populations behave like
financial portfolios in many ways---we can treat the abundance of
populations, such as salmon in streams, as financial stock value, and
groups of populations, such as salmon within a river catchment, as
portfolios. If a group of populations are stressed and react differently
then the risk of sudden decline of the group may be lowered, similar to
a diversified financial portfolio. This risk reduction has been referred
to as the `portfolio effect'. In this thesis I consider three
applications of portfolio concepts to ecology. I begin by evaluating
methods for estimating ecological portfolio effects and applying these
methods to moth, reef fish, and salmon metapopulations from around the
world. I show an inherent bias to a commonly used method, outline
recommendations for estimating ecological portfolio effects, and show
that the general tendency is for stabilizing effects. In the next
chapter, I use a portfolio approach to managing population diversity to
inform conservation priorities for salmon populations under a changing
climate. I show that preserving a diversity of thermal tolerances
minimizes risk given environmental stochasticity and ensures persistence
given long-term environmental change. However, this reduction in
variability can come at the expense of long-term persistence if climate
change increasingly restricts available habitat---forcing ecological
managers to balance society's desire for short-term stability and
long-term viability. Finally, I take the concept of black swans (extreme
and unexpected events) from the financial literature and ask what
evidence there is for these events across hundreds of bird, mammal,
insect, and fish abundance time series. I find strong but rare (3--5\%)
evidence of ecological black swans. Black swans are predominantly (87\%)
downward events and tend to be associated with some combination of
extreme climate and natural enemies (predators and parasites) with
little relationship to life history. My thesis demonstrates the
importance of conserving ecological properties that may contribute to
portfolio effects, such as thermal-tolerance diversity and habitat
heterogeneity, and highlights the importance of developing conservation
strategies that are robust to unexpected extreme events.

\bigskip
\noindent
\textbf{Keywords}: biodiversity, black swans, ecological portfolios,
extremes, diversity-stability, portfolio effects, population dynamics,
risk
